\lecture{4}{25 Sep. 14:20}{}

\section{Sorting Problem}

\begin{note}
quick sort
\end{note}

\begin{note}
half sort sort
\end{note}

\subsection{排序問題下界}
解決了排序問題的下界就可以一次解決
\begin{itemize}
    \item The (worst-case) time complexity of the comparison-based sorting problem is $\Omega(n\log n)$.
    \item The $O(n \log n)$-time analysis for the Half-Sort-Sort algorithm is tight.
    \item Learning {Reduction}
\end{itemize}

\begin{definition}[Permutation Problem]
    For the instance
    \begin{itemize}
        \item Input: An array $A$ of $n$ distinct integers.
        \item Output: Reorder the n-index array $B = [1, 2, \cdots, n]$ such that \[
        A[B[1]] < A[B[2]] < \cdots < A[B[n]].
        \]
    \end{itemize}
\end{definition}
\vspace{1em}

排列難度 $\leq$ 排序難度。
If the comparison-based sorting problem can be solved in $O(f(n))$ time, then so can the comparison-based permutation problem.

\section{Amortized Analysis}