\chapter{Complexity for a Problem}
\lecture{2}{11 Sep. 13:20}{}

\section{函數成長率(Rate of Growth)}

\begin{exercise}[棋癡國王與文武大臣]
    國王愛下棋,文武大臣要獎賞
    \begin{itemize}
        \item 武大臣每下一個棋子,獎賞多一袋米,起始為一袋米
        \item 文大臣每下一個棋子,獎賞雙倍,起始為一粒米
    \end{itemize}
\end{exercise}
\begin{answer}
    棋盤 64 格
    \begin{itemize}
        \item 武大臣:$n$ 袋米
        \item 文大臣:$2^{n}$ 粒米
    \end{itemize}
    $2^n$ 的成長率遠遠高於 $n$,單位的影響不及成長率
\end{answer}

\section{成長率的比較}

\begin{note}
    雖然 200 年前就有 Asymptotic Notation 的概念,但直到 1970 年代才被演算法分析之父 Donald Ervin Knuth 正式定義到 CS 領域內。
\end{note}

\begin{exercise}[Why Asymptotic Notation]
    為什麼要用 Asymptotic Notation?
\end{exercise}
\begin{answer}
    問題難度通常單位不一致
    \begin{itemize}
        \item $n=3$ 魔方問題要 20 轉
        \item $n$ 個信封的老大問題要 $n-1$ 次比較
    \end{itemize}
    兩者難度無法比較
\end{answer}

\begin{definition}[Rate of Growth]
    \blue{沒有人有明確定義},但是成長率很好比較,有很多東西也是無法定義但可以比較,e.g. 無限集合可以比大小。
\end{definition}

\newpage

\section{Big Oh Notation}

\begin{definition}[Big Oh Notation]\label{df:1.3.1}
    For functions $f, g : \mathbb{N} \rightarrow \mathbb{R}$, we write
    \[
        f(n) = O(g(n))
    \]
    to satisfy the extistence of positive constants $c$ and $n_0$ such that the inequality
    \[
        0 \leq f(n) \leq c \cdot g(n)
    \]
    holds for all integer $n \geq n_0$.
\end{definition}

\begin{note}
    $f(n), g(n)$ should be non-negative for sufficiently large $n$.
\end{note}

\vspace{1em}

The definition of \[
f(n) = O(g(n))
\]
says that there exist a positive constant $c$ such that the value of $f(n)$ is upper-bounded by $c \cdot g(n)$ for all sufficiently large positive $n$. \\

\begin{remark}
    因此 $O(g(n))$ 可以理解成一個成長率不高過 $g$ 的函數所 成的集合
\end{remark}

\subsection{等號左邊也有 Big-Oh}

\begin{definition}
    The equality $O(g(n)) = O(h(n))$ signifies that
    \[
    f(n) = O(h(n))
    \]
    holds for all functions $f(n)$ with
    \[
    f(n) = O(g(n))
    \]
    i.e. $O(g(n)) = O(h(n))$ signifies that $f(n) = O(g(n))$ implies $f(n) = O(h(n))$.\\

    The equality $=$ in $O(g(n)) = O(h(n))$ is more like $\subseteq$, i.e., $O(g(n)) \subseteq O(h(n))$.
\end{definition}

\begin{theorem}\label{thm:1.3.1}
    $O(g(n)) = O(h(n))$ if and only if $g(n) = O(h(n))$.
\end{theorem}
\begin{proof}
Consider the two directions separately.
\begin{itemize}
    \item For the $(\Rightarrow)$ case: 
    We can easily proof that \[
        g(n) = O(g(n))
    \]
    then we can deduce that \[
        \red{g(n)} = O(g(n)) = \red{O(h(n))}
    \]
    
    \item For the $(\Leftarrow)$ case: 
    \begin{prev}[\blue{Definition}~\ref{df:1.3.1}]
        \[
        g(n) = O(h(n)) \quad \Rightarrow \quad \exists \; c_1, n_1 > 0, \; \forall n \geq n_1, \; 0 \leq g(n) \leq c_1 \cdot h(n)
        \]
    \end{prev}

    Let $f$ be the function such that $f(n) = O(g(n))$. Then, by definition, we can deduce that
    \[
        \exists \; c_2, n_2 > 0, \; \forall n \geq n_2, \; 0 \leq f(n) \leq c_2 \cdot g(n).
    \]

    Assume $n \geq \max\{n_1, n_2\}$. Then, we have
    \[
        0 \leq f(n) \leq c_2 \cdot g(n) \leq c_2 \cdot (c_1 \cdot h(n)) = (c_1 c_2) \cdot h(n).
    \]

    Thus, we can conclude that \[
        f(n) = O(g(n)) = O(h(n))
    \]
\end{itemize}
    Hence, \[
        O(g(n)) = O(h(n)) \iff g(n) = O(h(n)).
    \]
\end{proof}

\section{Big-Oh 的運算}

\begin{exercise}
    所以,Big-Oh 相加的意思是什麼?
\end{exercise}

\begin{definition}[Big-Oh Addition]
    The equality $$O(g_1(n)) + O(g_2(n)) = O(h(n))$$ signifies that the equality
    \[
    f_1(n) + f_2(n) = O(h(n))
    \]
    holds for any functions $f_1(n)$ and $f_1(n)$ with
    \[
    f_1(n) = O(g_1(n))
    \]
    \[
    f_2(n) = O(g_2(n)).
    \]
    That is, $f_1(n) = O(g_1(n))$ and $f_2(n) = O(g_2(n))$ together imply $f_1(n) + f_2(n) = O(h(n))$.
\end{definition}
    

\begin{remark}
    雖然 $O(g_1(n)) + O(g_2(n))$ 看起來像是兩個集合的聯集,但相同集合想法\textbf{\red{無法帶到減乘除}}。
\end{remark}

\begin{definition}[Big-Oh $\circ$] The equality \[
        O(g_1(n))\circ O(g_2(n)) = O(h(n))
    \]\[
        g_1(n) \circ O(g_2(n)) = O(h(n))
    \]
    集合的復合操作
    \begin{notation}
        $$
        \{f_1(n)\circ f_2(n) \; | \; f_1(n) \in S_1 \text{ and } f_2(n) \in S_2\}
        $$
    \end{notation}
    可以被理解成
    \begin{itemize}
        \item 把 $=$ 解成 $\subseteq$
        \item 把 $g_1(n)$ 理解成 $\{g_1\}$
        \item $O(g1(n))$ 解為成長率不超過 $g_1$ 的成長率的所有函數所組成的集合
    \end{itemize}
\end{definition}

\begin{remark}
    減乘除應被理解成與剛剛加法類似的模式,而無法被理解為集合的運算
\end{remark}

\begin{definition}[Big-Oh $-,\ \cdot,\ /$]
    (Take $-$ as the example) The equality \[ O(g_1(n)) - O(g_2(n)) = O(h(n)) \] signifies the equality \[
        f_1(n) - f_2(n) = O(h(n))
    \]
    holds for any functions $f_1(n)$ and $f_2(n)$ with
    \[
        f_1(n) = O(g_1(n))
    \]
    \[
        f_2(n) = O(g_2(n))
    \]
\end{definition}

\begin{exercise}
    Proof or disproof: \[
        O(n)^{O(\log_2 n)} = O(2^n)
    \]
\end{exercise}
\begin{answer}
    First, we take $\log$ on both sides:
    \[
        \text{LHS} = O(\log n) \cdot O(\log n) = (O(\log n))^2
    \]    
    \[
        \text{RHS} = O(n)
    \]
    LHS grows slower than RHS, therefore the original statement is true.
\end{answer}
\begin{remark}
    $\log$ 的底數不影響成長率,因此可忽略。
\end{remark}

\begin{definition}[Big-Oh套Big-Oh]
    The equality \[
        O(O(g(n))) = O(h(n))
    \]
    signifies that the equality
    \[
        O(f(n)) = O(h(n))
    \]
    holds for any function $f$ with \[
        f(n) = O(g(n))
    \]
    i.e. $f(n) = O(g(n))$ implies $O(f(n)) = O(h(n))$.
\end{definition}

\begin{theorem}
    $g(n) = O(h(n))$ if and only if $O(O(g(n))) = O(h(n))$
\end{theorem}
\begin{proof}
    Consider the two directions separately.
    \begin{itemize}
        \item For the $(\Rightarrow)$ case: 
        \begin{prev}[\blue{Definition}~\ref{df:1.3.1}]
            \[
            g(n) = O(h(n)) \ \Longrightarrow \ \exists \; c_0, n_0 > 0, \; \forall n \geq n_0, \; 0 \leq g(n) \leq c_0 \cdot h(n)
            \]
        \end{prev}
        \(f(n) = O(O(g(n)))\) signifies that for \( c_1, c_2, n_1, n_2 > 0 \)
        \[
            \forall n \geq n_1, \; 0 \leq f(n) \leq c_2 \cdot u(n); \quad \forall n \geq n_2, \; 0 \leq u(n) \leq c_1 \cdot g(n)
        \]
        Get all together, we have
        \[
            \red{0 \leq f(n)} \leq c_2 \cdot (c_1 \cdot g(n))\  \red{\leq c_2 c_1 c_0 \cdot h(n)} \quad \Longrightarrow \quad f(n) = O(h(n))
        \]
        Thus, we can conclude that \[
            O(O(g(n))) = O(h(n))
        \]

        \item We can easily proof that \[
            g(n) \subseteq O(g(n)) \subseteq O(O(g(n)))
        \]
        Then we can get \[
            g(n) = O(O(g(n))) = O(h(n))
        \]
    \end{itemize}
    Hence, \( g(n) = O(h(n)) \iff O(O(g(n))) = O(h(n))\)
\end{proof}

\begin{theorem}[Rules of Computation in Big-Oh]
    The following statements hold for functions $f,g: \mathbb{N} \rightarrow \mathbb{R}$ such that there is a constant $n_0$ such that $f(n)$ and $g(n)$ for any integer $n \geq n_0$:
    \begin{itemize}
        \item \textbf{Rule 1}: $f(n) = O(f(n))$.
        \item \textbf{Rule 2}: If $c$ is a positive constant, then $c \cdot f(n) = O(f(n))$.
        \item \textbf{Rule 3}: $f(n) = O(g(n))$ if and only if $O(f(n)) = O(g(n))$.
        \item \textbf{Rule 4}: $O(f(n)) \cdot O(g(n)) = O(f(n) \cdot g(n))$.
        \item \textbf{Rule 5}: $O(f(n) \cdot g(n)) = f(n) \cdot O(g(n))$
    \end{itemize}
\end{theorem}
\begin{proof}
    For \textbf{Rule 5}: By the \blue{Definition~\ref{df:1.3.1}}, $u(n) = O(f(n) \cdot g(n))$ signifies that there exist positive constants $c_1$ and $n_1$ such that the inequality 
    \[
        \exists \; c_0, n_0 > 0,\  \forall n \geq n_0,\ 0 \leq u(n) \leq c_0 \cdot f(n) \cdot g(n)
    \]
    the definition of $u(n) = f(n) \cdot O(g(n))$ is 
    \[
        \exists \; c_1, n_1 > 0,\  \forall n \geq n_1,\ 0 \leq u(n) \leq f(n) \cdot c_1 \cdot g(n)
    \]
    which are equivilence to each other.
\end{proof}

\section{More Asymptotic Notation}

\begin{definition}[Little-oh]
    For any function $f, g : \mathbb{N} \rightarrow \mathbb{R}$, we write
    \[
        f(n) = o(g(n))
    \]
    to signify that for any constant $c > 0$, there is a positive constant $n_0(c)$ such that
    \[
        0 \leq f(n) < c\cdot g(n)
    \]
    holds for each integer $n \geq n_0(c)$

    \begin{note}
        \(n_0(c)\) is a function of $c$. When we \(n_0(c)\) is a constant, we means that it does not depend on $n$.
    \end{note}
\end{definition}

\vspace{1em}

白話來說 $f(n) = o(g(n))$ 的定義是說,不管是多小的常數 $c$,要 $n$ 夠大(i.e., $n \geq n_0(c)$),
\[
0 \leq f(n) < c \cdot g(n)
\]
都還是成立。

\begin{eg}
$$n = o(n^2)$$
\end{eg}
\vspace{0.5em}
Observe that for any positive constant $c$, as long as $n > \frac{1}{c}$, we have 
\[
0 \leq n < c\cdot n^2
\]
Therefore, we may let $n_0(c) = \frac{1}{c} + 1$ and have $n = o(n^2)$ proved.

\begin{definition}[Other notation]
    The other notation can be defined via $O$ and $o$ notation:
    \begin{itemize}
        \item We write $f(n) = \Omega(g(n))$ if \[
            g(n) = O(f(n)).
        \]
        \item We write $f(n) = \Theta(g(n))$ if \[
            f(n) = O(g(n)) \text{ and } f(n) = \Omega(g(n))
        \]
        \item We write $f(n) = \omega(g(n))$ if \[
            g(n) = o(n)
        \]
    \end{itemize}
\end{definition}


Limit notation 可以幫我們判斷各種 Asymptotic Notation:
\begin{itemize}
    \item If \[
        \lim_{n \rightarrow \infty} \frac{f(n)}{g(n)} = 0
    \], the we can guess $f(n) = o(g(n))$.
    \item If \[
        \lim_{n \rightarrow \infty} \frac{f(n)}{g(n)} = c
    \], the we can guess $f(n) = \Theta(g(n))$.
    \item If \[
        \lim_{n \rightarrow \infty} \frac{f(n)}{g(n)} = \infty
    \], the we can guess $f(n) = \omega(g(n))$.
\end{itemize}
\vspace{1em}

然而,極限不一定應可以推至 Asymptotic Notation:
\begin{itemize}
    \item Let $f(n) = g(n) = (-1)^n$. We have \[
        \lim_{n \rightarrow \infty} \frac{f(n)}{g(n)} = 1,
    \] but $f(n) \neq O(g(n))$, $f(n) \neq \Omega(g(n))$, and $f(n) \neq \Theta(g(n))$.
    \item Let $f(n) = (-1)^n$ and $g(n) = n \cdot (-1)^n$. We have \[
        \lim_{n \rightarrow \infty} \frac{f(n)}{g(n)} = 0,
    \] but $f(n) \neq o(g(n))$.
    \item Let $f(n) = 2 + (-1)^n$ and $g(n) = 2-(-1)^n$. We have \[
        f(n) = \Theta(g(n)),
    \] but $\lim_{n\rightarrow \infty} \frac{f(n)}{g(n)}$ does not exist.
\end{itemize}

\begin{exercise}
    Can we just use $\leq$ instead of $<$ in the definition of $o$ ? 
\end{exercise}
\begin{answer}
    In most part of it will be right. However there will be a special situation:
    \[
        o(0) = 0
    \]
    which is definetly wrong.
\end{answer}

\begin{exercise}
    為何不都用 $\exists \; c_0, n_0$ 或都用 $\forall c, n_0(c)$ ?
\end{exercise}
\begin{answer}
    如果都用 $\exists \; c_0, n_0$,那$o$就會退化,變成$O$而已,並且$<$, $\leq$是沒有太大差別的
    \vspace{1em}
    \begin{proof}
        Suppose that $\hat{n}_0(c)$ is the constant ensured by the $\leq$-version. 
        We simply let $$n_0(c) = \max(m_0,\hat{n}_0(c/2)).$$ 
        As a result, for any positive constant $c$, if $n \geq n_0(c)$, we have $g(n) > 0$ and thus
        \begin{align*}
            0 < f(n) &\leq \frac{c}{2} \cdot g(n) \\
                    &< c \cdot g(n).
        \end{align*}
    \end{proof}
證畢,由此可知符號並無太大影響,不可讓$o$退化
\end{answer}

\newpage