\lecture{3}{18 Sep. 14:20}{}

\section{問題難度的比較}


如果,
\begin{itemize}
    \item $P$ 不比 $Q$ 難 且
    \item $Q$ 不比 $P$ 簡單
\end{itemize}
那兩個問題的難度相同

\begin{definition}
    We say that the (worst-case) time complexity of Problem $P$ is $\Theta(f(n))$ if
    \begin{itemize}
        \item the time complexity of Problem $P$ is $O(f(n))$, i.e.
            \[
                \text{there \red{exists} an }O(f(n))\text{-time algorithm that solves Problem }P
            \]
        \item the time complexity of Problem $P$ is $\Omega(f (n))$, i.e. 
            \[
                \text{\red{any} algorithm that solves Problem }P\text{ requires }\Omega(f (n))\text{ time (in the worst case).}
            \]
        對於任何演算法,只要存在一組 instance 可以達成,一組 $\Omega(f(n))$ 即可推出
    \end{itemize}
\end{definition}

\begin{note}
若沒有特別說,$n$代表的是 input(instance) size,儲存資料所需的容量
\end{note}

\begin{note}
「正確的演算法」就是對於所有合法輸入都可以對應出正確的輸出,的解決問題方法
\end{note}

\section{演算法複雜度比較}

\section{分析演算法複雜度下界}

\begin{note}
    儘管有些 case 可以,但 Big-Omega 不可以跟 Big-Oh 一樣分析
\end{note}

\begin{remark}
    \redbox{$\Omega$-time 必須要一組一組 instance 分析}
\end{remark}

\chapter{演算法的設計與分析}

\section{Half Sorted}
\begin{definition}[Half Sorting Problem]
    An $n$-element array $A$ is half-sorted if
    \[
        A[i] \leq A\left[ \left\lfloor \frac{i}{2} \right\rfloor \right]
    \]
    holds for each index $i$ with $2 \leq i \leq n$.
\end{definition}

\textbf{Half-sorting Problem}:
\begin{itemize}
    \item Input: \[
        \text{An array A of n distinct numbers.}
    \]
    \item Output: \[
        \text{A half-sorted array that is reordered from A.}
    \]
\end{itemize}

\begin{note}
歸約 Reduction(問題重整)
\end{note}

\begin{note}
把問題的性質變強,便可以順便證明性質較弱的問題
\end{note}

\begin{definition}[順調法]
    
\end{definition}

算上面算比較少,樹的上下是不對稱的

