\chapter{Introduction}
\lecture{1}{4 Sep. 14:20}{}

\section{Design and Analysis}

\subsection{Design}

\begin{remark}
    \blue{\textbf{Find the point to cut into the problem.}}
\end{remark}
\begin{exercise}[Coffee and Milk]
        把 500 毫升的咖啡倒入 10 毫升,再從 510 毫升牛奶咖啡取 10 毫升倒入 490 毫升牛奶中,試問兩邊比例?
\end{exercise}
\begin{answer} 
    兩邊都固定 500 毫升,一邊少的必定出現在另一邊,\textbf{切入點對了根本不用計算}
\end{answer}

\subsection{Analysis}

\begin{exercise}[Card]
        把牌洗亂(平均)需要幾次?
\end{exercise}
\begin{answer}
    \begin{note}
        \textbf{定義何為亂?}\\
        排列出現機率皆為
        \[
        \frac{1}{52!}
        \]
    \end{note}
        七次是充分必要條件(嚴謹分析 on paper)$n$ card should shuffle $\frac{3}{2}\log_2 n + \theta$ times.
\end{answer}

\begin{definition}[亂]
    With $n$-cards, we have to let the probbability of every combination become 
    \[
    \frac{1}{n!}
    \]
\end{definition}

\newpage

\begin{exercise}[Top-in shuffle]
    Consider Top-in shuffle with the cards. How to get it "randomly" ?
\end{exercise}
\begin{answer}
    Define the $k$-th section to be 初始底牌從底下數上來是 $k$-th card.

    \begin{enumerate}
        \item bottom $k-1$ cards must be 亂
        \item 每次都可以用 $n/k$ 次將他洗亂,因為出現機率皆為 $k/n$
    \end{enumerate}

    We can shuffle $n \cdot H_n$ times.
\end{answer}

\begin{theorem}
    底下 $k-1$ 張卡片永遠是亂的
\end{theorem}
\begin{proof}
    考慮 top-in shuffle,利用數學歸納法
    \begin{itemize}
        \item 第一輪要插入底牌下方,只有 1 個空隙,因此必須插入,因此插入的機率是 \[
        \frac{1}{1!}
        \]
        \item 底下如果有 k 張牌,假設下面 k 張是亂的,表示他的排列 $k!$ 種,每種順序機率都是 \[
        \frac{1}{k!}
        \]
        \item 再插入一張,共有 $k+1$ 個空隙,排起來每種順序出現的機率為 \[
        \frac{1}{(k+1)} \cdot \frac{1}{k!} = \frac{1}{(k+1)!}
        \]
        符合亂的定義
    \end{itemize}
\end{proof}

第 $k$ 階段插入到下面都是從 $n$ 個空隙裡面找到 $k$ 個空隙插入,因此出現機率必定為 \(\frac{k}{n}\),因此需要 shuffle 次數為\[
\frac{n}{k}
\]
接著考慮第 $n$ 階段,底牌不是亂的,因此要再洗一次,因此最終的和為\[
\sum_{i=1}^{n} \frac{n}{i} = n \cdot \sum_{i=1}^{n}\frac{1}{i} = n\cdot H_n
\]

\begin{note}
    choose another card to be "bottom", 可以減少第一次的$1/n$ 就可以少 $n/1$ 次 shuffle. 因此可以把次數減少為:
    \[
        n\cdot H_n - n
    \]
\end{note}

\begin{remark}
     \blue{\textbf{簡單的分析點交換就可以造成巨大的影響}}
\end{remark}

\newpage

\section{Jargons}
\begin{definition}[Problems]
    「問題」(Problem)是一個對應關係,就是一個函數
    \begin{itemize}
        \item 演算法核心是在探討問題的解決難易度
        \item 有些問題確定很難,就不用妄想想出簡單演算法
    \end{itemize}
\end{definition}

\begin{definition}[Instance]
    「個例」(instance),也就是問題的合法輸入
\end{definition}

\begin{definition}[Computation Model]
    「計算模型」(Computation Model),也就是遊戲規則,同個問題在不同的規則下可能難易度不同

    \begin{itemize}
        \item Comparision base \& Computation base
    \end{itemize}
    
\end{definition}

\begin{definition}[Algorithm]
    「演算法」Algorithm is a detail step-by-step instruction
    \begin{itemize}
        \item 符合規則
        \item 詳細步驟
    \end{itemize}
\end{definition}

\begin{definition}[Hardness]
    「難度」(Hardness),想知道一個「\textbf{問題}」有多難解,用最厲害的一個「\textbf{解法}」,對於每個「\textbf{個例}」,都至少要用多少「\textbf{工夫}」才能解完
\end{definition}

\begin{itemize}
    \item 魔方問題:對於所有解法,存在至少一個初始 instance 讓解法需要 20 次才能轉完,切入點是找到一個固定的初始狀態,這是一個已經最佳化的問題
\end{itemize}

\begin{theorem}[Confirm Hardness]
    用 \textbf{upper bound} 和 \textbf{lower bound} 去夾起來決定難度
    \begin{itemize}
        \item 當 \textbf{upper bound} = \textbf{lower bound} 的時候,我們才知道問題的確切難度
        \item 有些情況,就算夾起來也不一定可以確定難度
    \end{itemize}
\end{theorem}
\begin{proof}
    \[
    \begin{matrix}
        \boxed{\text{upper bound}}\\
        \Downarrow \\
        \\
        \\
        \\
        \Uparrow \\
        \boxed{\text{lower bound}}
    \end{matrix}\quad\implies\quad
    \begin{matrix}
        \boxed{\text{upper bound}}\\
        \Downarrow\\
        \Uparrow \\
        \boxed{\text{lower bound}}
    \end{matrix}
    \]
\end{proof}

\begin{note}
    我們在這門課都討論 worst case
\end{note}