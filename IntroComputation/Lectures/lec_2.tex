\lecture{2}{2025-09-08}{}

\section{Nondeterministic Finite Automata (NFA)}

First, we see a NFA that accept strings with 1 in 3rd position from the end,

\begin{figure}[H]
    \centering
    \begin{tikzpicture}[->, node distance=2cm, >=Stealth, on grid, auto]
    \node[state,initial] (q_1) {$q_1$};
    \node[state] (q_2) [right of=q_1] {$q_2$};
    \node[state] (q_3) [right of=q_2] {$q_{3}$};
    \node[state,accepting] (q_4) [right of=q_3] {$q_{4}$};    
    
      \path (q_1) edge[loop above] node {$0,1$} (q_1)
            (q_1) edge[above]  node {$1$} (q_2)
            (q_2) edge[above]  node {$0,\epsilon$} (q_3)
            (q_3) edge[above]  node {$1$} (q_4)
            (q_4) edge[loop below] node {$0,1$} (q_4);
    \end{tikzpicture}
    \caption{NFA machine}
\end{figure}

\begin{itemize}
    \item $\delta$ is not a function, i.e. $\delta(q_1, 1) = q_1 \text{ or } q_2$
    \item $\epsilon$ between $q_2, q_3$ means $q_2$ can move to $q_3$ without any input
\end{itemize}

\newpage

We can transport NFA to DFA by some method, for example, for the above NFA we can have:

\begin{figure}[H]
    \centering
    \begin{tikzpicture}[->, node distance=3cm, >=Stealth, on grid, auto]
    \node[state, initial] (000) {$q_{000}$};
    \node[state, accepting, right of=000] (100) {$q_{100}$};
    \node[state, right of=100] (010) {$q_{010}$};
    \node[state, accepting, right of=010] (110) {$q_{110}$};
    \node[state, below of=000] (001) {$q_{001}$};
    \node[state, accepting, right of=001] (101) {$q_{101}$};
    \node[state, right of=101] (011) {$q_{011}$};
    \node[state, accepting, right of=011] (111) {$q_{111}$};

    \draw (000) edge[loop above] node{$0$} (000)
    (000) edge[left] node{$1$} (001)
    (100) edge[above] node{$0$} (000)
    (100) edge[above] node{$1$} (001)

    (010) edge[bend right=10,above] node{$1$} (101)
    (010) edge[above] node{$0$} (100)

    (110) edge[bend right,above] node{$0$} (100)
    (110) edge[above] node{$1$} (101)

    (001) edge[bend right,below] node{$1$} (011)
    (001) edge[above] node{$0$} (010)

    (101) edge[bend right=10,below] node{$0$} (010)
    (101) edge[above] node{$1$} (011)

    (011) edge[above] node{$1$} (111)
    (011) edge[above] node{$0$} (110)

    (111) edge[loop right, below] node{$1$} (111)
    (111) edge[right] node{$0$} (110);

    \end{tikzpicture}
    \caption{NFA machine transport to DFA}
\end{figure}

We can record it in three bits, it will be complicated.

\begin{definition}[power set]
    \[
    P(Q) = \{ X | X \in Q \}
    \]
    which contain all the $2^{|Q|}$ combinations.
\end{definition}

\begin{definition}[Nondeterministic Finite Automata (NFA)]
    We define a NFA as a 5-tuple
    \[
        M = (Q, \Sigma_{\epsilon}, \delta, q_0, F)
    \]
    where
    \begin{itemize}
        \item $Q$: Set of states (\red{Finite})
        \item $\Sigma_{\epsilon} = \Sigma \cup \{\epsilon\}$
        \item $\delta$: $Q \times \Sigma_{\epsilon} \rightarrow P(Q)$
        \item $q_0 \in Q$
        \item $F \subset Q$
    \end{itemize}
\end{definition}

\begin{theorem}
    We have $w$
    \[
        w = y_1\cdots y_m \quad \text{where } y_i \in \Sigma_{\epsilon}
    \]
    A sequence $r_0\cdots r_m$ such that
    \begin{enumerate}[label=(\arabic*)]
        \item $r_0 = q_0$
        \item $r_{i+1} = \delta(r_i, y_{i+1}),\quad i = [0, n-1]$
        \item $r_n \in F$
    \end{enumerate}
\end{theorem}

\begin{note}
    So $m$ may not be the original length (as $y_i$ may be $\epsilon$)
\end{note}

\subsection{Equivalence of DFA and NFA}

From DFA $\Rightarrow$ NFA. Formally DFA is not an NFA due to $\Sigma$ and $\Sigma_{\epsilon}$. but we can easily handle this by adding
\[
    q_i, \epsilon \rightarrow \emptyset
\]

For NFA $\Rightarrow$ DFA, we have the example on the slides on a graph.


\begin{figure}[H]
    \centering
    \begin{tikzpicture}[->, node distance=3cm, >=Stealth, on grid, auto]
        
        \node[state, initial, accepting] (q1) {$q_1$};
        \node[state, below left of=q1] (q2) {$q_2$};
        \node[state, below right of=q1] (q3) {$q_3$};
        
        \draw (q1) edge[left] node{$b$} (q2)
        (q2) edge[loop left, left] node{$a$} (q2)
        (q1) edge[bend right, left] node{$\epsilon$} (q3)
        (q2) edge[below] node{$a, b$} (q3)
        (q3) edge[bend right, right] node{$a$} (q1);

    \end{tikzpicture}
    \caption{NFA example}
\end{figure}

\[
    \Downarrow
\]

\begin{figure}[H]
    \centering
    \begin{tikzpicture}[->, node distance=3cm, >=Stealth, on grid, auto]
        
        \node[state, initial, initial where=left, accepting] (13) {$\{1, 3\}$};
        \node[state, below of=13] (2) {$\{2\}$};
        \node[state, right of=13] (3) {$\{3\}$};
        \node[state, right of=3] (phi) {$\emptyset$};
        \node[state, right of=2] (23) {$\{2, 3\}$};
        \node[state, accepting, right of=23] (123) {$\{1, 2, 3\}$};
        \draw
        (phi) edge[loop above] node{$a, b$} (phi)
        (2)
        edge[above] node{$a$} (23)
        (2)
        edge[above] node{$b$} (3)
        (3)
        edge[above] node{$b$} (phi)
        (3)
        edge[above] node{$a$} (13)
        (13) edge[loop above] node{$a$} (13)
        (13) edge[left] node{$b$} (2)
        (23) edge[bend left, above] node{$a$} (123)
        (23) edge[right] node{$b$} (3)
        (123) edge[loop right] node{$a$} (123)
        (123) edge[bend left, below] node{$b$} (23);

    \end{tikzpicture}
    \caption{DFA convertion example}
\end{figure}

\begin{itemize}
    \item Remove the states that are not reachable.
    \item Remove the states that not handle the $\epsilon$ transition. For example, the start state
    \[
        \{ q_1 \} \ \red{\text{wrong}} \quad \rightarrow \quad \{ q_1, q_3 \} \ \text{\grn{correct}}
    \]
\end{itemize}

\begin{definition}
    \[
    E(\{ q_0 \}) = \{q_0\} \cup \{\text{states reached by $\epsilon$ from $q_0$}\}
    \]
\end{definition}

\vspace{2em}

Then we can redefine the procedure formally.

\newpage

\begin{theorem}
    Given a NFA
    \[
        M = (Q,\Sigma, \delta, q_0, F)
    \]
    We can convert it to a DFA
    \[
        M' = (Q', \Sigma, \delta', q'_0, F')
    \]
    where 
    \begin{itemize}
        \item $Q' = P(Q)$
        \item $q'_0 \in P(Q) = E(\{q_0\})$
        \item $F' = \{ R \mid R \in Q', R \cap F \neq \emptyset \}$
        \item $\delta'$: 
        \[
            \delta'(R, a) = \bigcup_{r \in R} E(\delta(r, a))
        \]
    \end{itemize}
\end{theorem}

\subsection{Closure under reqular operations}

We give two NFAs $N_1, N_2$, 
\begin{align*}
    &N_1 =(Q_1, \Sigma, \delta_1, q_1, F_1) \\
    &N_2 =(Q_2, \Sigma, \delta_2, q_2, F_2)
\end{align*}
note that $\epsilon \notin \Sigma$, and the graph of them are:


\begin{figure}[H]
    \centering
    \begin{tabular}{llllll}
        $N_1$ &&&&& $N_2$\\
    \begin{tikzpicture}[node distance=3cm, every node/.style={scale=0.5}]
    \node[state, initial] (1) {$q_1$};
    \node[state, above right of=1, yshift=-1.4cm] (2) {};
    \node[state, below right of=1, yshift=1.4cm] (3) {};
    \node[state, right of=1, xshift= 1.4cm] (6) {};
    \node[state, accepting, above right of=6, yshift=-1cm] (4) {};
    \node[state, accepting, below right of=6, yshift=1cm] (5) {};
    \end{tikzpicture}
        &&&&& \begin{tikzpicture}[node distance=3cm, every node/.style={scale=0.5}]
    \node[state, initial] (01) {$q_2$};
    \node[state, above right of=01, yshift=-1.4cm] (02) {};
    \node[state, below right of=01, yshift=1.2cm] (03) {};
    \node[state, right of= 01, xshift= 1cm] (06) {};
    \node[state, accepting, above right of=06, yshift=-1cm] (04) {};
    \node[state, accepting, below right of=06, yshift=1cm] (05) {};
    \node[state, accepting, right of=06, xshift=1.7cm] (04) {};
    \end{tikzpicture}
    \end{tabular}
    \caption{$N_1, N_2$}
\end{figure}

\begin{itemize}
    \item \textbf{Union}: We can contrruct the $N_1 \cup N_2$ in
    
    
    \begin{figure}[H]
        \centering
        \begin{tikzpicture}[node distance=3cm, every node/.style={scale=0.5}, ->, >=Stealth, on grid, auto]
        \node[state, initial] (0) {$q_0$};    
        \node[state, above right of=0] (1) {$q_1$};
        \node[state, above right of=1, yshift=-1.4cm] (2) {};
        \node[state, below right of=1, yshift=1.4cm] (3) {};
        \node[state, right of=1, xshift= 1.4cm] (6) {};
        \node[state, accepting, above right of=6, yshift=-1cm] (4) {};
        \node[state, accepting, below right of=6, yshift=1cm] (5) {};
        \node[state, below right of=0] (01) {$q_2$};
        \node[state, above right of=01, yshift=-1.4cm] (02) {};
        \node[state, below right of=01, yshift=1.2cm] (03) {};
        \node[state, right of= 01, xshift= 1cm] (06) {};
        \node[state, accepting, above right of=06, yshift=-1cm] (04) {};
        \node[state, accepting, below right of=06, yshift=1cm] (05) {};
        \node[state, accepting, right of=06, xshift=1.7cm] (04) {};
    
        \path (0) edge[above] node {$\epsilon$} (1);
        \path (0) edge[below] node {$\epsilon$} (01);
        \end{tikzpicture}
    \caption{$N_1 \cup N_2$}
    \end{figure}

    \begin{proposition}[Construction of Union]
        New NFA is
        \[
            N_1 \cup N_2 = (Q,\; \Sigma,\; \delta,\; q_0,\; F)
        \] 
        where
        \begin{itemize}[label=$\circ$]
            \item $Q = Q_1 \cup Q_2 \cup \{ q_0 \}$
            \item $\delta: $
            \[
                \delta(q, a) = \begin{cases}
                    \delta_1(q, a) & q \in Q_1 \\
                    \delta_2(q, a) & q \in Q_2 \\
                    \{q_1, q_2 \} & q = q_0, a = \epsilon \\
                    \emptyset & q = q_0, a \neq \epsilon
                \end{cases}
            \]
            \item $F = F_1 \cup F_2$
        \end{itemize}
    \end{proposition}

    \item \textbf{Concatenation}: We can construct the $N_1 \circ N_2$ in
    
    \begin{figure}[H]
        \centering
        \begin{tikzpicture}[node distance=3cm, every node/.style={scale=0.5}, ->, >=Stealth, on grid, auto]
        \node[state, initial] (1) {};
        \node[state, above right of=1, yshift=-1.4cm] (2) {};
        \node[state, below right of=1, yshift=1.4cm] (3) {};
        \node[state, right of=1, xshift= 1.4cm] (6) {};
        \node[state, above right of=6, yshift=-1cm] (4) {};
        \node[state, below right of=6, yshift=1cm] (5) {};
        \node[state, right of=6, xshift=2cm] (01) {};
        \node[state, above right of=01, yshift=-1.4cm] (02) {};
        \node[state, below right of=01, yshift=1.2cm] (03) {};
        \node[state, right of= 01, xshift= 1cm] (06) {};
        \node[state, accepting, above right of=06, yshift=-1cm] (04) {};
        \node[state, accepting, below right of=06, yshift=1cm] (05) {};
        \node[state, accepting, right of=06, xshift=1.7cm] (04) {};

        \path (4) edge[above] node {$\epsilon$} (01);
        \path (5) edge[below] node {$\epsilon$} (01);
        \end{tikzpicture}
    \caption{$N_1 \circ N_2$}
    \end{figure}

    \begin{proposition}[Construction of Concatenation]
        New NFA is
        \[
            N_1 \circ N_2 = (Q,\; \Sigma,\; \delta,\; q_0,\; F)
        \] 
        where
        \begin{itemize}[label=$\circ$]
            \item $Q = Q_1 \cup Q_2$
            \item $\delta: $
            \[
                \delta(q, a) = \begin{cases}
                    \delta_1(q, a) & q \in Q_1 \ F_1 \\
                    \delta_2(q, a) & q \in Q_2 \\
                    \delta_1(q, \epsilon) \cup \{ q_2 \} & q \in F_1, a = \epsilon \\
                    \delta_1(q, \epsilon)& q \in F_1, a \neq \epsilon \\
                \end{cases}
            \]
            \item $q_0 = q_1$
            \item $F =F_2$
        \end{itemize}
    \end{proposition}
    
    \item \textbf{Kleene star}: $N_1^*$ can also accept $\{ \emptyset \}$, then we can construct the $N_1^*$ in

    \begin{figure}[H]
        \centering
        \begin{tikzpicture}[node distance=3cm, every node/.style={scale=0.5}, ->, >=Stealth, on grid, auto]
        \node[state, initial, accepting] (0) {};  
        \node[state, right of=0] (1) {};
        \node[state, above right of=1, yshift=-1.4cm] (2) {};
        \node[state, below right of=1, yshift=1.4cm] (3) {};
        \node[state, right of=1, xshift= 1.4cm] (6) {};
        \node[state, above right of=6, accepting, yshift=-1cm] (4) {};
        \node[state, below right of=6, accepting, yshift=1cm] (5) {};

        \path (4) edge[bend right=35, above] node {$\epsilon$} (1);
        \path (5) edge[bend left=35, below] node {$\epsilon$} (1);
        \path (0) edge[above] node {$\epsilon$} (1);
        \end{tikzpicture}
    \caption{$N_1^*$}
    \end{figure}

    \begin{proposition}[Construction of Kleene Star]
        New NFA is
        \[
            N_1^* = (Q_1,\; \Sigma,\; \delta_1,\; q_0,\; F_1)
        \] 
        where
        \begin{itemize}[label=$\circ$]
            \item $Q = Q_1 \cup \{q_0\}$
            \item $\delta: $
            \[
                \delta(q, a) = \begin{cases}
                    \delta_1(q, a) & q \in Q_1 \ F_1 \\
                    \delta_1(q, a) \cup \{q_1\} & q \in F_1, a = \epsilon \\
                    \delta_1(q, \epsilon) & q \in F_1, a \neq \epsilon \\
                    \{q_1\} & q = q_0, a = \epsilon \\
                    \emptyset & q = q_0, a \neq \epsilon
                \end{cases}
            \]
            \item $F =F_1 \cup \{q_0\}$
        \end{itemize}
    \end{proposition}    
\end{itemize}

\begin{note}
    Some operations are also closed under regular languages,
    \begin{itemize}[label=$\circ$]
        \item \textbf{Intersection:}
        \[
            A_1 \cap A_2
        \]
        Use the product automaton (the same construction as for Union).
        A string is accepted if and only if the state is in the accept states of both $N_1$ and $N_2$ at the same time.

        \item \textbf{Set Difference:}
        \[
            A_1 - A_2
        \]
        Use the product automaton as well.
        A string is accepted if the state is in the accept states of $N_1$ but \emph{not} in the accept states of $N_2$.

        \item \textbf{Complement:}
        \[
            A_1^c = \Sigma^* - A_1
        \]
        Since $\Sigma^*$ is regular and the class of regular languages is closed under set difference,
        $A_1^c$ is also regular.
    \end{itemize} 
\end{note}