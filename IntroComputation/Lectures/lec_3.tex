\lecture{3}{2025-09-15}{}

\section{Regular expressions}

A regualar expression is a tool to describe a language.

\begin{definition}[Regular expressions]
    $R$ is a regular expressions if it is one of the following expressions: 
    \begin{enumerate}[label=(\arabic*)]
        \item $a$, where $a \in \Sigma$
        \item $\varepsilon$ $(\varepsilon \notin \Sigma)$
        \item $\emptyset$
        \item $R_1 \cup R_2$,  where $R_1, R_2$ are regular expressions
        \item $R_1 \circ R_2$, where $R_1, R_2$ are regular expressions
        \item $R_1^*$,  where $R_1$ is a regular expression
    \end{enumerate}
\end{definition}

If their is no parentheses, we follow the order of:
\[
\boxed{\texttt{Kleene star}} \rightarrow \boxed{\texttt{Concatenation}} \rightarrow \boxed{\texttt{Union}}
\]

\newpage

\begin{remark}
    \[
        R^+ = RR^*, \quad R^+ \cup \{\varepsilon\} = R^*
    \]
\end{remark}

For $\emptyset$ and $\varepsilon$, we have 
\begin{itemize}
    \item $\varepsilon$: empty string
    \item $\emptyset$: empty language (language without any string)
    \begin{align*}
        &(0 \cup \varepsilon)1^*=01^* \cup 1^* \\
        &(0 \cup \emptyset)1^*=01^* \\
        &\emptyset 1^* = 1^* \emptyset = \emptyset
    \end{align*}
    
\end{itemize}


\begin{eg}
Here are some examples,
\begin{itemize}
    \item 
    Strings that start and end with the same symbol:
    \begin{equation*}
        0 \Sigma^*0 \cup 1 \Sigma^*1\cup 0 \cup 1
    \end{equation*}
    \item $(\Sigma \Sigma)^*$: strings with even length
    \item $R\cup \emptyset=R$
    \item $R \circ \varepsilon=R$
    \item $\emptyset^*=\{\varepsilon\}$
\end{itemize}
\end{eg}

Floating point numbers can also be represented by regular expressions. For example,
\[
(+\cup-\cup\varepsilon)(DD^*\cup DD^*.D^*\cup D^*.DD^*), \text{ where } D = \{0, \ldots,9\}
\]

\begin{eg}
\begin{align*}
    & 72 \in DD^*\\
    & 2.1 \in DD^*.D^*\\
    &  7. \in DD^*.D^*\\
    & .01 \in D^*.DD^*
\end{align*}
\end{eg}

\begin{lemma}
    Language by a regular expression $\Longrightarrow$ Regular (described by an automaton)
\end{lemma}
\begin{proof}
    The proof is by induction,
    \begin{itemize}
        \item $R = a \in \Sigma$ can be recognize by
        \begin{center}
            \begin{tikzpicture}[node distance=2cm, ->, >=Stealth, on grid, auto]
                \node[state, initial] (q1) {};
                \node[state, accepting, right of=q1] (q2) {};
    
                \draw (q1) edge[above] node{$a$} (q2);
            \end{tikzpicture}
        \end{center}
        \begin{align*}
            & N = (\{q_1,q_2\}, \Sigma, \delta, q_1, \{q_2\}) \\
            & \delta(q_1,a)= \{q_2\} \\
            & \delta(r,b)=\emptyset, r \neq q_1\mbox{ or } b \neq a
        \end{align*}

        \item $R = \varepsilon$
        \begin{center}
            \begin{tikzpicture}[node distance=2cm, ->, >=Stealth, on grid, auto]
                \node[state, initial, accepting] (q1) {};
            \end{tikzpicture}
        \end{center}
        \begin{align*}
            & N=(\{q_1\},\Sigma,\delta, q_1, \{q_1\}) \\
            & \delta(q_1,a)=\emptyset, \forall a
        \end{align*}

        \item $R = \emptyset$
        \begin{center}
            \begin{tikzpicture}[node distance=2cm, ->, >=Stealth, on grid, auto]
                \node[state, initial] (q1) {};
            \end{tikzpicture}
        \end{center}
        \begin{align*}
            &N=(\{q\}, \Sigma, \delta, q, \emptyset)\\
            &\delta(r,a)=\emptyset, \forall r, a
        \end{align*}

        \item $R=R_1\cup R_2, \ R=R_1 \circ R_2, \ R=R_1^*$ have proof by NFA.
    \end{itemize}
\end{proof}

\subsection{Convert a DFA to a regular expression}

The idea is:
\begin{enumerate}[label=$\arabic*^\circ$]
    \item DFA $\longrightarrow$ GNFA
    \item Remove states from GNFA until only the start and accept states.
\end{enumerate}

\begin{exercise}
Convert the following DFA into regular expression.
\begin{center}
    \begin{tikzpicture}[node distance=2.5cm, ->, >=Stealth, on grid, auto]
        \node[state, initial] (1) {1};                    
        \node[state, accepting, right of=1] (2) {2};
        \node[state, accepting, below right of=1] (3) {3};              

        \draw (1) edge[bend left=10, above] node{$a$} (2);
        \draw (2) edge[bend left=10, below] node{$a$} (1);
        \draw (1) edge[bend left=10, right] node{$b$} (3);
        \draw (3) edge[bend left=10, below] node{$b$} (1);    
        
        \draw (2) edge[loop above] node{$b$} (2);
        \draw (3) edge[right] node{$a$} (2);        
    \end{tikzpicture}
\end{center}
\end{exercise}
\begin{answer}
First, convert to GNFA:
\begin{center}
\begin{tikzpicture}[node distance=2.5cm, ->, >=Stealth, on grid, auto]
    \node[state, initial] (s) {$s$};

    \node[state, above right of=s] (1) {1};                    
    \node[state, right of=1] (2) {2};
    \node[state, below right of=1] (3) {3};              
    \node[state, accepting, below right of=2] (a) {$a$};        

    \draw (1) edge[bend left=10, above] node{$a$} (2);
    \draw (2) edge[bend left=10, below] node{$a$} (1);
    \draw (1) edge[bend left=10, right] node{$b$} (3);
    \draw (3) edge[bend left=10, below] node{$b$} (1);    

    \draw (2) edge[loop above] node{$b$} (2);
    \draw (3) edge[right] node{$a$} (2);        

    \draw (s) edge[above] node{$\varepsilon$} (1);        

    \draw (2) edge[above] node{$\varepsilon$} (a);        
    \draw (3) edge[below] node{$\varepsilon$} (a);
\end{tikzpicture}
\end{center}

Next, is to remove the states one by one. We skip, so we can get the answer:
\[
(a(aa\cup b)^* ab \cup b)
((ba \cup a)(aa \cup b)^* ab \cup bb)^*((ba \cup a)
    (aa\cup b)^* \cup \varepsilon) \cup a(aa \cup b)^*
\]
which is very complicated.
\end{answer}

\begin{definition}[Generalized NFA(GNFA)]
    We define a GNFA as a 5-tuple
    \[
        G = (Q, \Sigma, \delta, q_{start}, q_{accept})
    \]
    where
    \begin{itemize}
        \item $F$ is not a se, but a single accept state $q_{accept}$
        \item $\delta$ function is:
        \[
            (Q-\{q_{accept}\}) \times (Q-\{q_{start}\})\rightarrow R
        \]
        where $R$ is all regular expressions over $\Sigma$.
        \item Two new states:
        \begin{align*}
            &q_{start} \rightarrow q_0 \text{ with }\varepsilon \\
            &\text{any } q \in F \rightarrow q_{accept}\text{ with }\varepsilon
        \end{align*}
    \end{itemize}
\end{definition}

Consider $q_{\text{rip}}$ is the state being removed
\begin{center}
    \begin{tikzpicture}[node distance=2.5cm, ->, >=Stealth, on grid, auto]
    \node[state] (qi) {$q_i$};
    \node[state, right of=qi] (qj) {$q_j$};
    \node[state, below right of=qi] (qr) {$q_{\text{rip}}$};  

    \draw (qi) edge[bend right, left] node{$R_1$} (qr);
    \draw (qi) edge[bend left, above] node{$R_4$} (qj);  
    \draw (qr) edge[bend right, right] node{$R_3$} (qj);
    \draw (qr) edge[loop below, right] node{$R_2$} (qr);            
    \end{tikzpicture}
\end{center}

The new regular expression between $q_i$ and $q_j$ is

\begin{center}
\begin{tikzpicture}[node distance=2.5cm, ->, >=Stealth, on grid, auto]
  \node[state] (qi) {$q_i$};
  \node[state, right of=qi, xshift=3cm] (qj) {$q_j$};        

  \draw (qi) edge[above] node{$(R_1)(R_2)^*(R_3)\cup (R_4)$} (qj);          
  \end{tikzpicture}
\end{center}

We can wrote the whole process into a algorithm.

\begin{algorithm}[H]
\caption{\textsc{convert}$(G)$ — State-Elimination from GNFA to RE}
\KwIn{$G=(Q,\Sigma,\delta,q_s,q_a)$ a GNFA}
\KwOut{A regular expression $R$ for the language of $G$}
\BlankLine

$k \gets |Q|$\;  \tcp*{number of states}

\If{$k=2$}{
    \Return $\delta(q_s,q_a)$ \tcp*{the (single) edge label from $q_s$ to $q_a$}
}

Choose any $q_{\mathrm{rip}} \in Q \setminus \{q_s,q_a\}$\;
$Q' \gets Q \setminus \{q_{\mathrm{rip}}\}$\;
Initialize $\delta'$ as the restriction of $\delta$ to $Q' \times Q'$\;
\BlankLine

\ForEach{$q_i \in Q' \setminus \{q_a\}$}{
    \ForEach{$q_j \in Q' \setminus \{q_s\}$}{
        $R_1 \gets \delta(q_i, q_{\mathrm{rip}})$\;
        $R_2 \gets \delta(q_{\mathrm{rip}}, q_{\mathrm{rip}})$\;
        $R_3 \gets \delta(q_{\mathrm{rip}}, q_j)$\;
        $R_4 \gets \delta(q_i, q_j)$\;
        $\delta'(q_i, q_j) \gets R_4 \;\cup\; \big(R_1\, R_2^{*}\, R_3\big)$\;
    }
}
\BlankLine

$G' \gets (Q', \Sigma, \delta', q_s, q_a)$\;
\Return \textsc{convert}$(G')$\;

\end{algorithm}