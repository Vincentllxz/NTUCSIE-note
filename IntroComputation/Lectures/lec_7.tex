\newpage
\section{Deterministic Pushdown Automata}
\lecture{7}{2025-11-03}{}

PDA is non-deterministic in general. However, there is a special class of PDA called \red{Deterministic} Pushdown Automata (DPDA). From Ch.1 we know
\[
    \text{DFA} \equiv \text{NFA}
\]
but
\[
    \text{DPDA} \neq \text{PDA} \ \Longrightarrow \text{CFL} \neq \text{DCFL}
\]

\begin{definition}[Deterministic Pushdown Automaton (DPDA)]
    A deterministic pushdown automaton (DPDA) is a 6-tuple \[
    M = (Q, \Sigma, \Gamma, \delta, q_0, F)
    \] where
    \begin{itemize}
        \item \(Q\): States
        \item \(\Sigma\): Input alphabet
        \item \(\Gamma\): Stack alphabet
        \item \(\delta\): Transition function
        \[
            Q \times \Sigma_{\varepsilon} \times \Gamma_{\varepsilon} \to (Q \times \Gamma_{\varepsilon}) \cup \{ \emptyset \}
        \]
        \item \(q_0 \in Q\): Start state
        \item \(F \subset Q\): Set of accepting states
    \end{itemize}
\end{definition}

To build a DPDA, we first look at the different between PDA and DPDA.
\begin{prev}
    For PDA, 
    \[
        \delta: Q \times \Sigma_{\varepsilon} \times \Gamma_{\varepsilon} \to \mathcal{P}(Q \times \Gamma_{\varepsilon})
    \]
\end{prev}
\begin{note}
    In DPDA, for \(\forall q \in Q, a \in \Sigma, x, \gamma \in \Gamma\), at most and at least one of the following is true:
    \[
        \delta(q, a, x) = (p, \gamma), \quad \delta(q, a, \varepsilon) = (p, \gamma), \quad \delta(q, \varepsilon, x) = (p, \gamma), \quad \delta(q, \varepsilon, \varepsilon) = (p, \gamma)
    \]
    the rest must be \(\emptyset\).
\end{note}

\subsection{Acceptance, Rejection of DPDA}

The Rejection of DPDA is similar to PDA, which should only happen when
\begin{itemize}
    \item Not end at an accept state after the last symbol.
    \item DPDA fails to read the input
    \begin{enumerate}
        \item \texttt{pop} an empty stack
        \item Endless $\varepsilon$-transition
    \end{enumerate}
\end{itemize}

\newpage

\begin{eg}
    $L = \{ 0^n 1^n \mid n \geq 0\}$
\end{eg}

\begin{figure}[H]
    \centering
    \begin{tikzpicture}[node distance=2.5cm, ->, >=Stealth, auto]
    \node[state,initial,accepting] (q_1) {$q_1$};
    \node[state] (q_2) [right of=q_1] {$q_2$};
    \node[state] (q_3) [below of=q_2] {$q_{3}$};
    \node[state,accepting] (q_4) [left of=q_3] {$q_{4}$};      
    \path 
    (q_1) edge[above]  node {$\epsilon, \epsilon \rightarrow \$ $} (q_2)
    (q_2) edge[loop right]  node {$0,\epsilon \rightarrow 0$} (q_2)
    (q_2) edge[right]  node {$1, 0 \rightarrow \epsilon$} (q_3)
    (q_3) edge[loop right]  node {$1, 0 \rightarrow \epsilon$} (q_3)  
    (q_3) edge[below] node {$\epsilon, \$ \rightarrow \epsilon$} (q_4);
    \end{tikzpicture}
    \caption{DPDA for \(L = \{ 0^n 1^n \mid n \geq 0\}\)}
\end{figure}

The Transition function is defined as follows:
\begin{center}
    
    \begin{tabular}{l|ccc|ccc|ccc}
    &
    \multicolumn{3}{c|}{0} &
    \multicolumn{3}{c|}{1} &
    \multicolumn{3}{c}{$\epsilon$}\\ \hline
    & 0 & \$ & $\epsilon$ 
    & 0 & \$ & $\epsilon$ 
    & 0 & \$ & $\epsilon$ \\ \hline
    $q_1$ &$\emptyset$&$\emptyset$&$\emptyset$&$\emptyset$&$\emptyset$&
    $\emptyset$&$\emptyset$&$\emptyset$& $(q_2,\$)$\\
    $q_2$ &$\emptyset$&$\emptyset$&$(q_2,0)$&$(q_3,\epsilon)$&$q_r$&
    $\emptyset$&$\emptyset$&$\emptyset$& $\emptyset$\\
    $q_3$ &$q_r$&$\emptyset$&$\emptyset$&$(q_3,\epsilon)$&$\emptyset$&$\emptyset$&$\emptyset$&$(q_4,\epsilon)$&$\emptyset$ \\
    $q_4$ &$q_r$&$q_r$&$\emptyset$&$q_r$&$q_r$&$\emptyset$&$\emptyset$&$\emptyset$&$\emptyset$ \\ 
    $q_r$ &$q_r$&$q_r$&$\emptyset$&$q_r$&$q_r$&$\emptyset$&$\emptyset$&$\emptyset$&$\emptyset$ \\ 
    \end{tabular}
\end{center}

To find this transition table, for instance, 
\begin{itemize}
    \item consider the state \(q_1\):
    \[
        \delta(q_1, \varepsilon, \varepsilon) = (q_2, \$)
    \]
    then we can implies that
    \[
        \delta(q_1, a, \gamma) = \delta(q_1, a, \varepsilon) = \delta(q_1, \varepsilon, \gamma) = \emptyset, \quad \forall a \in \Sigma = \{0,1\},\ \gamma \in \Gamma = \{0, \$\}
    \]
    \item consider the state \(q_2\):
    \[
        \delta(q_2, 1, 0) = (q_3, \varepsilon)
    \]
    then we can implies that
    \begin{center}
        \begin{tabular}{l|ccc|ccc|ccc}
        &
        \multicolumn{3}{c|}{0} &
        \multicolumn{3}{c|}{1} &
        \multicolumn{3}{c}{$\epsilon$}\\ \hline
        & 0 & \$ & $\epsilon$ 
        & 0 & \$ & $\epsilon$ 
        & 0 & \$ & $\epsilon$ \\ \hline
        $q_2$ &$\emptyset$&$\emptyset$&$(q_2,0)$&$(q_3, \epsilon)$& $\red{\neq \emptyset}$ &$\emptyset$
        &$\emptyset$&$\emptyset$& $\emptyset$\\
        \end{tabular}
    \end{center}
    due to \[
        \delta(q_2, 1, \varepsilon) = \delta(q_2, \varepsilon, 0) = \delta(q_2, \varepsilon, \varepsilon) = \emptyset
    \]
    Formally we have \[
        \delta(q_2, 1, \$) = (q_r, \varepsilon)
    \]
\end{itemize}

For the string \(011\), the computation of the DPDA is as follows:
\begin{equation*}
    q_1 \xrightarrow{\epsilon} q_2, \{\$\} \
q_1 \xrightarrow{0} q_2, \{0,\$\} \
q_1 \xrightarrow{1} q_3, \{\$\}\
q_1 \xrightarrow{\epsilon} q_4, \emptyset
\end{equation*}
Follow the graph,
\[
    \delta(q_4, 1, \epsilon) \text{ and } \delta(q_4, \epsilon, \epsilon) = \emptyset
\]
hence, the DPDA rejects \(011\).