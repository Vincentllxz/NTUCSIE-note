\chapter{Basic Knowledge}
\lecture{1}{2025-09-01}{}

\section{Mathematical Notions}

\subsection{Set \& its operation}

\begin{definition}[Set]
    Omitted    
\end{definition}

\begin{definition*}[Sequence \& Tuple]
Here are some definitions of basic containers

\begin{definition}[Sequence]
    Sequence is the objects in order, which have two properties:
    \begin{itemize}
        \item Order: \[
        (1, 2, 3) \neq (2, 1, 3)
        \]
        \item Repetition: \[
        \text{Sequence}:(1, 2, 3) \neq (1, 1, 2,3)
        \]
        \[
        \text{Set}: \{1, 2, 3\} = \{1, 1, 2, 3\}
        \]
    \end{itemize}
\end{definition}

\begin{definition}[Tuple]
    Finite sequence, \((1, 2, 3)\) is a 3-tuple
\end{definition}
    
\end{definition*}

\begin{definition}[Cartesian Product]
    Here is the Cartesian Product between two sets. We define
    \[
    A = \{1, 2\}, \ B = \{x, y\}
    \]
    then,
    \[
    A \times B = \{(1, x),(1, y),(2, x),(2, y)\}
    \]
\end{definition}

\newpage

\subsection{Function \& Relation}

\begin{definition}[Function]
    Function is a machine with single output.
\end{definition}


\begin{definition*}[Equivalence Relations]
Here are the properties of Equivalence Relations.
\begin{definition}[reflexive]
    \[
    \forall x, \ xRx
    \]
\end{definition}
\begin{definition}[symmetric]
    \[
    \forall x,y, \ xRy \ \Longleftrightarrow \ yRx
    \]
\end{definition}
\begin{definition}[transitive]
    \[
    xRy,\ yRz \ \Longrightarrow \ xRz
    \]
\end{definition}    
\end{definition*}

\begin{eg}
    \[
    i \equiv_{7} j, \ \text{ if }\ 0 = i - j \mod 7 
    \]
\end{eg}
\begin{itemize}
    \item Reflexive
    \[
    i-i = 0 \mod 7
    \]
    \item Symmetric
    \[
    i-j = 7a, \ j-i = -7a
    \]
    \item Transitive
    \[
    i - j = 7a,\ j - k = 7b \ \Longrightarrow \ i - k = 7(a + b)
    \]
\end{itemize}

\subsection{String \& Languages}

\begin{definition*}[String \& Languages]
    Here is the definition of Language.
    \begin{eg}[Alphabet]
        \[
        \{0, 1\}
        \]
    \end{eg}
    \begin{eg}[String]
        \[
        01000
        \]
    \end{eg}
    \begin{definition}[Language]
        Set of Strings
        \[
        L(A)
        \]
        is the language of $A$
    \end{definition}
\end{definition*}

\newpage

\section{Definitions, Theorems, and Proofs}

\begin{itemize}
    \item \textbf{Definition}: Introduce new concept.
    \item \textbf{Statement}: A sentence that is either true or flase.
    \item \textbf{Theorem}: A statement that is true.
    \begin{itemize}
        \item \textbf{Lemma}: A ``helping'' theorem.
        \item \textbf{Corllary}: A theorem that follows easily from another theorem.
    \end{itemize}
\end{itemize}

\subsection{Proof by Construction}
\begin{proposition}
    Sum of degrees of every graph is even
\end{proposition}
\begin{proof}
    Each edge contributes 2 nodes, so
    \[
    \sum_{v \in V} \deg(v) = 2 \times |E|
    \]
    Hence, the sum of degrees of every graph is even.
\end{proof}

\begin{note}
    The implication is the definition of graphs.
\end{note}

\subsection{Proof by Contradiction}

Assume the statement is false, then deduce a contradiction.

\subsection{Proof by Induction}

\begin{itemize}
    \item Basis: Prove for $n=0$ or $n=1$ or some trivial case.
    \item Inductive Step: Assume true for $n=k$ (Induction Hypothesis), prove for $n=k+1$.
\end{itemize}

\chapter{Regular Languages}

\section{Deterministic Finite Automata (DFA)}

\begin{itemize}
    \item Automaton: single
    \item Automata: plural
\end{itemize}

\begin{definition}[Deterministic Finite Automata (DFA)]
    We define a DFA as a 5-tuple
    \[
    (Q,\, \Sigma,\, \delta,\, q_0,\, F)
    \] 
    where
    \begin{itemize}
        \item $Q$: Set of states (\red{Finite})
        \item $\Sigma$: Alphabet (i.e. set of input characters) (\red{Finite})
        \item $\delta$: $Q \times \Sigma \rightarrow Q$: Transition Function
        \item $q_0 \in Q$: Start state
        \item $F \subset Q$: Set of accept states
    \end{itemize}
\end{definition}

\vspace{3em}

\begin{figure}[H]
    \centering
    \begin{tikzpicture}[node distance=3.5cm, ->, >=Stealth, on grid, auto]
      \node[state,initial] (q_1) {$q_1$};
      \node[state,accepting, right of=q_1] (q_2)  {$q_2$};
      \node[state, right of=q_2] (q_3)  {$q_3$};
    
      \draw (q_1) edge[loop above]    node {$0$} (q_1)
            (q_1) edge[above]  node {$1$} (q_2)
            (q_2) edge[loop above]    node {$1$} (q_2)
            (q_2) edge[bend left, above] node {$0$} (q_3)
            (q_3) edge[bend left, below] node {$0,1$} (q_2);
    \end{tikzpicture}
    \caption{A state diagram}
\end{figure}

If we call this machine $M$, then we have.

\[
M = (Q, \Sigma, \delta, q_0, F)
\]

\newpage

For the example given above,

\begin{gather*}
    Q=\{q_1,q_2, q_3\}\\
    \Sigma=\{0,1\}\\
    q_0 = q_1 \\
    F=\{q_2\}
\end{gather*}


The $\delta$ function:
\begin{center}
\begin{tabular}{c|cc}
    & 0 & 1\\ \hline
    $q_1$ & $q_1$ & $q_2$\\
    $q_2$ & $q_3$ & $q_2$\\
    $q_3$ & $q_2$ & $q_2$
\end{tabular}
\end{center}

\begin{definition}
    The language that recognize by a Machine $M$ is denoted as \[
        L(M) = A
    \]
    We say $A$ is recognizeed (accepted) by $M$.
\end{definition}

\subsection{Definition of Computation}

Let, 
\begin{itemize}
    \item $M = (Q,\ \Sigma,\ \delta,\ q_0,\ F)$ be a finite automaton.
    \item $w = w_1, \cdots, w_n$ be a string over $\Sigma$.
\end{itemize}

\begin{theorem}
    $M$ accepts $w$ if $\exists$ states $r_0 \cdots r_n$ such that
    \begin{enumerate}[label=(\arabic*)]
        \item $r_0 = q_0$
        \item $r_{i+1} = \delta(r_i, w_{i+1}),\quad i = [0, n-1]$
        \item $r_n \in F$
    \end{enumerate}
\end{theorem}

\begin{definition}[Regular Language]
    A language is regular if recognized by some automata.
\end{definition}

\subsection{Regular Operations}

\begin{definition*}
Assume $A, B$ are given languages,
\begin{definition}[Union]
    \[
        A \cup B = \{ w \;|\; w \in A \lor w \in B \}
    \]
\end{definition}
\begin{definition}[Concatenation]
    \[
        A \circ B = \{ w_1w_2 \;|\; w_1 \in A, w_2 \in B \}
    \]
\end{definition}
\begin{definition}[Kleene Star]
    \[
        A^* = \{ w_1 \cdots w_k \;|\; k \geq 0, w_i \in A \}
    \]
    which can also be defined as
    \[
        \bigcup_{i=1}^\infty A_i = \{\epsilon\} \cup A \cup A^2 \cup A^3 \cup \cdots, \quad A^0 = \{\epsilon\}, \ A^n = \{wv \;|\; w \in A^{n-1}, v \in A\}
    \]
\end{definition}
\end{definition*}

\begin{definition}[closed]
    We say an operation $R$ is closed if the following property holds if
    \[
        x \in A, y \in A, \text{ then } xRy \in A
    \]
\end{definition}

\begin{theorem}
    Regular languages are closed under the union, concatenation, and Kleene star.
\end{theorem}
\begin{proof}
We define two machines as follows
\begin{eqnarray*}
    && M_1 =(Q_1, \Sigma, \delta_1, q_1, F_1)\\
    && M_2 =(Q_2, \Sigma, \delta_2, q_2, F_2)\\
\end{eqnarray*}
if we union them, we can define a new machine
\[
M_1 \cup M_2 = \begin{cases}
    M = (Q, \Sigma, \delta, q_0, F) \\
    Q = \{ (r_1, r_2) \;|\; r_1 \in Q_1, r_2 \in Q_2 \}\\
    \delta((r_1, r_2), a) = (\delta_1(r_1, a), \delta_2(r_2, a))\\
    q_0 = (q_1, q_2) \\
    F = \{ (r_1, r_2) \;|\; r_1 \in F_1 \text{ or } r_2 \in F_2 \}
\end{cases}
\]
Hence, regular languages are closed under union.
\end{proof}