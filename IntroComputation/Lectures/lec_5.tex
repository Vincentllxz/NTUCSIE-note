\chapter{Context-Free Languages}

\lecture{6}{2025-10-20}{}

\section{Context-Free Grammars (CFG)}

Which is more powerful, and can be used in compilers. A \textbf{Grammar} is a collection of substitution rules that describe the structure of a language.

\begin{eg}
    Consider a grammar $G_1$:
    \begin{align*}
        A &\to 0A1 \\
        A &\to B \\
        B &\to \#
    \end{align*}
\end{eg}

Here are the jargon terms:
\begin{itemize}
    \item Each of one is called a \textbf{substitution rule}.
    \item \textbf{Variables} (non-terminals): $A, B$ (Capital letters)
    \item \textbf{Terminals}: $0, 1, \#$ (Lowercase letters, numbers, symbols)
    \item \textbf{Start variable}: $A$ (the variable we start with)
\end{itemize}

The process of generating strings is called \textbf{derivation}. $G_1$ generates $000\#111$ by
\[
A \Rightarrow 0A1 \Rightarrow 00A11 \Rightarrow 000A111 \Rightarrow 000B111 \Rightarrow 000\#111
\]
We can show the derivation using a \textbf{parse tree}:

\begin{center}
    
    \begin{forest}
    for tree={
        parent anchor=south,
        child anchor=north,
        if n children=0{
          font=\itshape,
          tier=terminal,
        }{},
      }    
      [$A$
       [ 
        [$0$
        ]
       ]
       [$A$
        [
        [$0$
        ]
        ]
        [$A$
         [[$0$
         ]]
         [$A$
          [$B$
           [$\#$
           ]
          ]
         ]
         [[$1$
         ]]
        ]
        [[$1$
        ]]
       ] 
       [[$1$
       ]]
      ]
    \end{forest}
\end{center}

\newpage

\subsection{Definition of CFG}
The language of grammar $G$ is denoted by $L(G)$, for the language we discuss here,
\[
L(G_1) = \{ 0^n \# 1^n \mid n \geq 0 \}
\]

Now we give the formal definition of CFG.
\begin{definition}[Context-Free Grammar]
    We defined a CFG as a 4-tuple 
    \[
        G = (V, \Sigma, R, S)
    \]
    where
    \begin{itemize}
        \item $V$: Variables (Finite)
        \item $\Sigma$: Terminals (Finite)
        \item $R$: Rules:
        \[
            \text{Variables} \to \text{Strings of Variables and Terminals (including $\epsilon$)}
        \]
        \item $S \in V$: Start variable
    \end{itemize}
\end{definition}

For instance, for $G_1$,
\[
G_1 = (\{A, B\}, \{0, 1, \#\}, R, A)
\]
where $R$ is:
\[
    A \to 0A1 \mid B, \quad B \to \#
\]

\begin{notation}
    If $u, v, w$ are strings and rule $A \to w$ is applied, then we say
    \[
    uAv \ \text{yields}\ uwv
    \]
    denoted as
    \[
    uAv \Rightarrow uwv
    \]
\end{notation}

\begin{notation}
    If \[
    u = v \text{ or } u \Rightarrow u_1 \Rightarrow \cdots \Rightarrow u_k \Rightarrow v
    \] then we write
    \[
    v \xRightarrow{*} u
    \]
\end{notation}

\begin{definition}[Language of a CFG]
    The language generated by a CFG $G$ with start variable $S$ is
    \[
    L(G) = \{ w \in \Sigma^* \mid S \xRightarrow{*} w \}
    \]
\end{definition}

\newpage

\subsection{Examples of CFGs}

\begin{exercise}
    Consider the grammar $G_2 = (\{S\}, \{a, b\}, R, S)$:
    \[
        S \to aSb \mid SS \mid \epsilon
    \]
    What is $L(G_2)$?
\end{exercise}
\begin{answer}
If we let $a, b$ be the left and right parentheses respectively, then $L(G_2)$ is the set of all balanced parentheses.
\end{answer}

\begin{eg}
    Consider the grammar $G_3 = (V, \Sigma, R, S)$
    where
    \begin{itemize}
        \item $V = \{\langle \text{expr} \rangle, \langle \text{term} \rangle, \langle \text{factor} \rangle\}$
        \item $\Sigma = \{ +, \times, (, ), a \}$
        \item $R$: 
        \begin{align*}
            \langle \text{expr} \rangle &\to \langle \text{term} \rangle + \langle \text{expr} \rangle \mid \langle \text{term} \rangle \\
            \langle \text{term} \rangle &\to \langle \text{factor} \rangle \times \langle \text{term} \rangle \mid \langle \text{factor} \rangle \\
            \langle \text{factor} \rangle &\to ( \langle \text{expr} \rangle ) \mid a
        \end{align*}
    \end{itemize}
\end{eg}

First consider the string $a + a \times a$:




\newpage



\section{Chomsky Normal Form}

\newpage