\lecture{2}{9 Sep. 13:20}{}
\section{An Example of Gaussian Elimination}

\begin{eg}
Here is a linear equation.
\[
\begin{cases}
    2u + v + w &= 5\\
    4u - 6v &= -2\\
    -2u + 7v + 2w &= 9
\end{cases}
\]  
\end{eg}

\[
\left(
\begin{matrix}
    \boxed{\textbf{2}} & 1 & 1 & 5 \\
    4 & -6 & 0 & -2 \\
    -2 & 7 & 2 & 9
\end{matrix}
\right)
\Longrightarrow
\left(
\begin{matrix}
    2 & 1 & 1 & 5 \\
    0 & \boxed{\textbf{-8}} & -2 & -12 \\
    0 & 8 & 3 & 14
\end{matrix}
\right)
\Longrightarrow
\left(
\begin{matrix}
    2 & 1 & 1 & 5 \\
    0 & -8 & -2 & -12 \\
    0 & 0 & \boxed{\textbf{1}} & 2
\end{matrix}
\right)
\quad \text{"\textbf{pivot}"}
\]
Then we get \(w=2\), we can plug in the equation
i.e.
\[
\begin{cases}
    2u + v + 1w = 5 \\
      -8v  -2w = -12 \\
       w= 2
\end{cases}\quad \Longrightarrow \quad \text{\textcolor{blue}{Forward Elimination}}
\]
Then we substitute into 2nd, 1st equation to get \(v=1\) and \(u=1\) \(
\Longrightarrow
\) \textcolor{blue}{Backend Elimination}

\begin{note}
    By definition, \textcolor{red}{pivots cannot be zero}!
\end{note}

\begin{exercise}
    Under what circumstances could the elimination process break down?
\end{exercise}
\begin{answer}
    Here are some situations.
    \begin{itemize}
        \item Something \textcolor{blue}{\textbf{must}} go wrong in the singular case.
        \item Something \textcolor{blue}{\textbf{might}} go wrong in the nonsingular case.
    \end{itemize}

    A zero appears in a pivot position! 
    
    If in the process, there are nonzero pivots, then there's only one solution.
\end{answer}

\newpage

\begin{eg}
\[
\left(
\begin{matrix}
    2 & 1 & 1 & 5 \\
    4 & -6 & 0 & -2 \\
    -2 & 7 & 2 & 9
\end{matrix}
\right)
\]
\end{eg}

\begin{enumerate}[label=(\arabic*)]
    \item If $a_{11} = 0 \quad \Longrightarrow \quad \text{nonsingular}$
    \item If $a_{22} = 0 \quad \Longrightarrow \quad \text{nonsingular}$
    \item If $a_{33} = 1 \quad \Longrightarrow \quad \text{singular}$
\end{enumerate}

\begin{exercise}
    How many separate arithmetical operations does elimination require for $n$ equations in $n$ unknowns?
\end{exercise}
\begin{answer}
    For a single operation.
    \[
    \text{a single operation} = \text{each division \& each multiplication-subtraction}
    \]
\end{answer}

\begin{itemize}
    \item \textbf{FE}: 
    \[
    \underbrace{
    \begin{matrix}
        x & x & \cdots & x &= &x \\
        \vdots & \vdots & & & & \vdots\\
        x & x & \cdots & x &= &x \\
    \end{matrix}
    }_{n}
    \]

    \[
    n(n-1) + (n-1)(n-2) + \cdots +(1^2-1) = \frac{n^3-n}{3} \sim \frac{n^3}{3} \ \text{steps}
    \]

    \item \textbf{RHS}: \[
    (n-1) + (n-2) + \cdots + 1 = \frac{n(n-1)}{2} \sim \frac{n^2}{2} \ \text{steps}
    \]
    \item \textbf{BF}:
    \[
    1 + 2 + \cdots + n = \frac{n(n+1)}{2} \sim \frac{n^2}{2} \ \text{steps}
    \]
\end{itemize}

\newpage

\section{Matrix Notation and Matrix Multiplication}

\[
\begin{cases}
    2u + 4v + -2w &= 2\\
    4u + 9v -3w &= 8\\
    -2u -3v + 7w &= 10
\end{cases}
\quad \Longrightarrow \quad
u\left(\begin{matrix}
2 \\4 \\-2
\end{matrix}\right) + 
v\left(\begin{matrix}
4 \\9 \\-3
\end{matrix}\right) +
w\left(\begin{matrix}
-2 \\-3 \\7
\end{matrix}\right) = 
\left(\begin{matrix}
2 \\8 \\10
\end{matrix}\right)
\] 

We can rewrite it in the below form.

\[
\underset{\blue{\text{coefficient matrix}}}{A=\left(\begin{matrix}
2 & 4 & -2 \\
4 & 9 & -3 \\
-2 & -3 & 7
\end{matrix}\right)_{3\times3}}, \quad
\underset{\blue{\text{unknowns}}}{x = \left(\begin{matrix}
u \\v \\w
\end{matrix}\right)_{3\times1}}, \quad
\underset{\blue{\text{RHS}}}{b=\left(\begin{matrix}
2 \\8 \\10
\end{matrix}\right)_{3\times1}}\quad\Longrightarrow\quad
\underset{\blue{\text{solution}}}{x = \left(\begin{matrix}
-1 \\2 \\2
\end{matrix}\right)_{3\times1}}
\]

\[
\Large \boxed{Ax=b}
\]

\begin{definition}
    An $m\times n$ matrix, \textcolor{red}{$A_{m\times n}$ over $\mathbb{R}$}, is an array with \textcolor{red}{$m$ rows} and \textcolor{red}{$n$ columns} of \textcolor{red}{real numbers}, which can be written as 
\[
A=\left(\begin{matrix}
a_{11} & a_{12} & \cdots & a_{1n} \\
a_{21} & a_{22} & \cdots & a_{2n} \\
\vdots  \\
a_{m1} & a_{m2} & \cdots & a_{mn} \\
\end{matrix}\right), \ \text{where } a_{ij} \in \mathbb{R} , \ \begin{cases}
    i: \text{index of \textcolor{blue}{row}}\\
    j: \text{index of \textcolor{blue}{column}}
\end{cases}
\]
\end{definition}

\begin{itemize}
    \item $\boxed{m\times n}$ is called the \textbf{dimensions (size)} of $A$ $\Longrightarrow$ dimension of a $()_{3\times 5}$ is \textcolor{red}{$3\times 5$}
    
    \item $\boxed{a_{ij}}$ is called the \textbf{elements/entry/coefficient} of $A$
    
    \item \textbf{\textcolor{blue}{Addition}}: $A = (a_{ij})_{m\times n}$, $\quad B = (b_{ij})_{m\times n}$
    \[
    A+B = (a_{ij} + b_{ij})_{m\times n}
    \]

    \item \textbf{\textcolor{blue}{Multiplication}}: $A = (a_{ij})_{m\times n}$, $\quad B = (b_{ij})_{n\times r}$
    \[
    AB = (c_{ij})_{\textcolor{red}{m\times r}}\ , \quad \text{where }\ c_{ij} = \sum_{k=1}^{n} a_{i\textcolor{blue}{k}}\ b_{\textcolor{blue}{k}j}
    \]

    \item \textbf{\textcolor{blue}{Scalar Multiplication}}:
    \[
    \alpha A = (\alpha a_{ij})_{m \times n}
    \]
    
    \item
    \[
    A_{m\times n} \ x_{n\times1} = b_{m\times 1}
    \]
        
        In particular, if $$A_{1\times n}B_{n\times1} = \mathbf{v}\cdot \mathbf{w} = ()_{1\times1}.$$ 
        Then it's the \textcolor{blue}{\textbf{inner product}} of vector $\mathbf{v}$ and vector $\mathbf{w}$
    
    \end{itemize}
        
\newpage

\begin{eg}
\[
Ax = \left(\begin{matrix}
2 & 4 & -2\\
4 & 9 & -3\\
-2 & 3 & -7\\
\end{matrix}\right)\left(\begin{matrix}
-1 \\ 2 \\ 2
\end{matrix}\right) = \left(\begin{matrix}
2\cdot(-1) & 4\cdot(2) & -2\cdot(2)\\
4\cdot(-1) & 9\cdot(2) & -3\cdot(2)\\
-2\cdot(-1) & 3\cdot(2) & -7\cdot(2)\\
\end{matrix}\right) = \left(\begin{matrix}
2 \\ 8 \\ 22
\end{matrix}\right)
\]    
\end{eg}

\[
(-1)\left(\begin{matrix}
2 \\4 \\-2
\end{matrix}\right) + 
2\left(\begin{matrix}
4 \\9 \\3
\end{matrix}\right) +
2\left(\begin{matrix}
-2 \\3 \\7
\end{matrix}\right)
\]

\begin{enumerate}[label=(\arabic*)]
    \item by row: 3 inner product
    \item by column: a linear combination of 3 columns of $A$
\end{enumerate}

\begin{eg}[1A]
    \(Ax\) is a combination of columns of $A$
\begin{align*}
A_{m\times n} x_{n\times 1} &= 
\left(\begin{matrix}
A_1 | A_2|\cdots|A_n
\end{matrix}\right)\left(\begin{matrix}
x_1 \\ x_2 \\ \vdots \\ x_n
\end{matrix}\right) \\
&= x_1(A_1) + x_2(A_2) + \cdots + x_n(A_n) = \left(\sum_{j=1}^{n} a_{ij}\ x_{j}\right)_{m\times1}
\end{align*}
\end{eg}

\subsection{The Matrix Form of One Elimination Step}

\begin{definition*}[1B]
    Matrix form
    \begin{definition}
        zero matrix:
        \[
        O = \left(\begin{matrix}
        0 & \cdots & 0 \\
        \vdots & \ddots & \vdots \\
        0 & \cdots & 0 \\
        \end{matrix}\right)
        \]
    \end{definition}

    \begin{definition}
        identity matrix: 
        \[
        I = \left(\begin{matrix}
        1 & \cdots & 0 \\
         & \ddots &  \\
        0 & \cdots & 1 \\
        \end{matrix}\right) = I_n = I_{n\times n}; \quad \begin{cases}
            A_{m\times n} I_{n} = A_{m\times n} \\
            A_{m\times n} = A_{m\times n}I_{n}
        \end{cases}
        \]
    \end{definition}

    \begin{definition}
     elementary matrix (elimination matrix):
        \[
        E_{ij} = \left(\begin{matrix}
        1 & \cdots & 0 & \cdots & 0 \\
        0 & \ddots &  & & \vdots\\
        \vdots &  & \ddots & &\vdots \\
        \vdots &  & -\ell & \ddots& 0\\
        0 & \cdots & \underset{\text{\textcolor{red}{jth column}}}{0}  & \cdots & 1\\
        \end{matrix}\right)\begin{matrix}
            \\ \\ \\ \text{\textcolor{red}{\small ith row}} \\
        \end{matrix} \quad \blue{\ell: \text{multiplier}}
        \]

        \vspace{1em}

        \[
        E_{ij} \cdot A = \left(\begin{matrix}
         &  &  \\
        \cdots & \textcolor{blue}{-\ell} &  \cdots & 1 \\
        \\
         &  &  \\
        \end{matrix}\right)\left(\begin{matrix}
         &  &  \\
         &  &  \\
         &  &  \\
         &  &  
        \end{matrix}\right)\begin{matrix}
         &  &  \\
        \longleftarrow \ \text{i-th} & \Longrightarrow \ \textcolor{cyan!80!black}{(\text{i-th row}) + (-\ell)(\text{j-th column})}\\
        \longleftarrow \ \text{j-th} & \Longrightarrow \ \textcolor{red}{\text{create zero at } (i,j) \text{ position!}}\\
         &  &  
        \end{matrix}
        \]

    \end{definition}
\end{definition*}

\begin{eg}
    \[
        \underset{E_{21}}{\left(\begin{matrix}
        1 & 0 & 0 \\
        \textcolor{red}{-2} & 1 & 0 \\
        0 & 0 & 1 \\
        \end{matrix}\right)}\underset{A}{\left(\begin{matrix}
        2 & 4 & -2 \\
        4 & 9 & -3 \\
        -2 & -3 & 7 \\
        \end{matrix}\right)} = \underset{EA}{\left(\begin{matrix}
        2 & 4 & -2 \\
        0_{\textcolor{red}{21}} & 9 & -3 \\
        -2 & -3 & 7 \\
        \end{matrix}\right)}
        \]
\end{eg}

\begin{note}
Here is two properties
    \begin{enumerate}
        \item $Ax=b \quad \Longrightarrow \quad E_{ij} Ax = E_{ij} b$
        \item $E_{ij}A \neq AE_{ij}$
    \end{enumerate}
\end{note}

\subsection{Matrix Multiplication}
\begin{enumerate}[label=(\arabic*)]
    \item The $(i, j)$-th entry of $AB$ is the inner product of the \textbf{i-th} of $A$ and the \textbf{j-th} of $B$.

    \item Each column of $AB$ is the product of a matrix A and \textbf{a column of B}
    \begin{align*}
        \Longrightarrow \quad \text{column}\ j \ \text{of } AB &= A \text{ times \textbf{j-th} of }B \\
        &= \text{linear combination of \textbf{columns of }}A\\
        &= b_{1j}A_{\underset{\text{any numbers}}{\boxed{\bigcdot}}1} + b_{2j}A_{\bigcdot2} + \cdots + b_{nj}A_{\bigcdot n}
    \end{align*}
    \begin{eg}
    \[
    \underset{A_{2\times3}}{\left(
    \begin{matrix}
    3 & 1 & 1 \\
    2 & 0 & -1
    \end{matrix}
    \right)}\underset{B_{3\times3}}{\left(
    \begin{matrix}
    5 & 0 & 1 \\
    -1 & 0 & 1 \\
    2 & 1 & 2 \\
    \end{matrix}
    \right)} = \underset{C_{2\times3}}{\left(
    \begin{matrix}
    16 & 1 & 1 \\
    8 & 0 & -1
    \end{matrix}
    \right)}
    \]
    
    
    \end{eg}
    1st column of $AB = \left(
    \begin{matrix}
    16 \\ 8
    \end{matrix}
    \right) = 5\cdot\left(
    \begin{matrix}
    3 \\ 2
    \end{matrix}
    \right)+(-1)\cdot\left(
    \begin{matrix}
    1 \\ 0
    \end{matrix}
    \right)+2\cdot\left(
    \begin{matrix}
    1 \\ -1
    \end{matrix}
    \right)$ 

    \item Each row of $AB$ is a product of a row of A and a matrix $B$.
    \begin{align*}
    \Longrightarrow \quad \text{i-th row of }AB &= \textbf{ } \text{of } A \text{ times } B.\\
    &= \text{linear combination of \textbf{rows of }}B \\
    &= a_{i1}B_{1\bigcdot} + a_{i2}B_{2\bigcdot} + \cdots + a_{in}B_{n\bigcdot}
    \end{align*}
    
\end{enumerate}

\begin{theorem}
    Let $A, B$ and $C$ be matrices (possibly rectangular). Assume that their dimension permit them to be added and multiplied in the following theorem
    \begin{enumerate}[label=(\arabic*)]
        \item The matrix multiplication is associative
        \[
        (AB)C = A(BC)
        \]
        
        \item Matrix operations are distributive
        \[
        A(B+C) = AB + AC
        \]
        \[
        (A+B)C = AC + BC
        \]

        \item Matrix multiplication is \textcolor{red}{non}commutative
        \[
        AB \neq BA \quad \text{\textcolor{red}{in general}}
        \]

        \item Identity Matrix
        \[
        A_{n\times n} I_n = I_n A_{n\times n} = A_{n\times n}
        \]
    \end{enumerate}
\end{theorem}

\begin{eg}
\[
\underset{\textcolor{blue}{\text{21}}}{E} = \left(\begin{matrix}
    1 & 0 & 0\\
    \boxed{-2} & 1 & 0\\
    0 & 0 & 1
\end{matrix}\right), \quad 
\underset{\textcolor{blue}{\text{31}}}{F} = \left(\begin{matrix}
    1 & 0 & 0\\
    0 & 1 & 0\\
    \boxed{1} & 0 & 1
\end{matrix}\right), \quad 
\underset{\textcolor{blue}{\text{32}}}{G} = \left(\begin{matrix}
    1 & 0 & 0\\
    0 & 1 & 0\\
    0 & \boxed{-1} & 1
\end{matrix}\right)
\]
\end{eg}

\begin{enumerate}[label=(\arabic*)]
    \item \[
    \underset{\textcolor{blue}{\text{21}}}{E}\ \underset{\textcolor{blue}{\text{31}}}{F} = \left(\begin{matrix}
    1 & 0 & 0\\
    -2 & 1 & 0\\
    1 & 0 & 1
    \end{matrix}\right)\quad \boxed{=} \quad \underset{\textcolor{blue}{\text{31}}}{F}\ \underset{\textcolor{blue}{\text{21}}}{E} = \left(\begin{matrix}
    1 & 0 & 0\\
    -2 & 1 & 0\\
    1 & 0 & 1
    \end{matrix}\right)
    \]

    \item \[
    \underset{\textcolor{blue}{\text{21}}}{E} \ \underset{\textcolor{blue}{\text{32}}}{G} = \left(\begin{matrix}
    1 & 0 & 0\\
    -2 & 1 & 0\\
    0 & -1 & 1
    \end{matrix}\right)\quad \boxed{\neq} \quad \underset{\textcolor{blue}{\text{32}}}{G}\ \underset{\textcolor{blue}{\text{21}}}{E}
    \]

    \item \[
    \underset{\textcolor{blue}{\text{32}}}{G}\ \underset{\textcolor{blue}{\text{31}}}{F}\ \underset{\textcolor{blue}{\text{21}}}{E} = \underset{\textcolor{blue}{\text{"right order"}}}{\left(\begin{matrix}
    1 & 0 & 0\\
    -2 & 1 & 0\\
    3 & -1 & 1
    \end{matrix}\right)}\quad \boxed{\neq} \quad \underset{\textcolor{blue}{\text{21}}}{E}\ \underset{\textcolor{blue}{\text{31}}}{F}\ \underset{\textcolor{blue}{\text{32}}}{G} = \left(\begin{matrix}
    1 & 0 & 0\\
    -2 & 1 & 0\\
    1 & -1 & 1
    \end{matrix}\right)
    \]
\end{enumerate}
  
\begin{note}
    The product of lower triangular matrices is a lower triangular matrix.
\end{note}

\newpage