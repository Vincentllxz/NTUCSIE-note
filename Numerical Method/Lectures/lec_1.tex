\setcounter{chapter}{0}
\chapter{Floating-point systems}
\lecture{1}{2025.12.09}{}
\section{Floating-point basics}

This chapter is mainly based on the science of floating-point arithmetics which is based on the IEEE standard 754.

\subsection{Why learning floating-point operations?}

\begin{eg}
    A one-variable problem
    \[
        \min_{x} f(x) \quad \text{where } x \geq 0
    \]
\end{eg}

In the normal program, we should set an \red{upper bound} of $x$, or $x$ may be wrongly increased to $\infty$. We have to find the largest representable number in the
computer

\begin{eg}
    A ten-variable problem
    \[
        \min_{\mathbf{x}} f(\mathbf{x}) \quad \text{where } x_i \geq 0, i=1,2,\ldots,10
    \]
\end{eg}

We want to know how many are zeros, we may use 
\begin{minted}[linenos]{c++}
for (int i = 0; i < 10; i++)
    if (x[i] == 0) count++;
\end{minted}

But people say that don't do the comparison of floating-point
\begin{minted}[linenos]{c++}
double epsilon = 1.0e-12;
for (int i = 0; i < 10; i++)
    if (x[i] <= epsilon) count++;
\end{minted}

Which is better? How to chose \texttt{epsilon}? Can't do the comparison of floating-point? We need to understand the floating-point representation.

\subsection{Floating-point Format}
We know \texttt{float} (single precision): 4 bytes and \texttt{double} (double precision) 8 bytes in C/C++.

\begin{definition}[format]
    A floating-point system requires a base $\beta$, precision $p$, significand (mantissa) $d.d\ldots d$
\end{definition}