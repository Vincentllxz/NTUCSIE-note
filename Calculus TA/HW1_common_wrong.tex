\documentclass{article}
% basics
\usepackage{xeCJK}
\setCJKmainfont{Kaiti TC} % 新細明體
\setCJKmainfont{Kaiti TC}[AutoFakeBold=6 , AutoFakeSlant=.2]

\usepackage[letterpaper,top=2cm,bottom=2cm,left=3cm,right=3cm]{geometry}
\linespread{1.2}
\usepackage{amssymb}
\usepackage{amsmath}
\usepackage{graphicx}
\usepackage{amsthm}
\usepackage[dvipsnames]{xcolor}
\usepackage{xeCJK} % 啟用中文支持
\usepackage{circuitikz} % 電路圖
\usepackage{tcolorbox} % 用來加方框
\usepackage{algorithm}
\usepackage[noend]{algpseudocode}
\usepackage{enumitem}
\usepackage{float}
\usepackage{tikz}
\usetikzlibrary{trees}
\usepackage{hyperref}
\usetikzlibrary{arrows.meta,automata,positioning}

% === theorem environments with mdframed ===
\usepackage{thmtools}   % 提供 declaretheorem
\usepackage{mdframed}   % theorem style 加邊框

\declaretheoremstyle[
  headfont=\bfseries\sffamily\color{black}, bodyfont=\normalfont,
  mdframed={
    linewidth=2pt,
    rightline=false, topline=false, bottomline=false,
    linecolor=black,
    nobreak=false
  }
]{thmblueline}

\declaretheorem[style=thmblueline, numbered=no, name=Reason]{reason}

\declaretheoremstyle[
  headfont=\bfseries\sffamily\color{black}, bodyfont=\normalfont,
  mdframed={
    linewidth=2pt,
    rightline=false, topline=false, bottomline=false,
    linecolor=NavyBlue!60!black,
    nobreak=false
  }
]{thmgreenline}

\declaretheorem[style=thmgreenline, numbered=no, name=Note]{note}

\declaretheoremstyle[
  headfont=\bfseries\sffamily\color{red!70!black}, bodyfont=\normalfont,
  mdframed={
    linewidth=2pt,
    rightline=false, topline=false, bottomline=false,
    linecolor=red, backgroundcolor=red!3,
    nobreak=false
  }
]{thmredbox}

\declaretheorem[style=thmredbox, numbered=no, name=Wrong Ans]{wrong}


\declaretheoremstyle[
  headfont=\bfseries\sffamily\color{ForestGreen!70!black}, bodyfont=\normalfont,
  mdframed={
    linewidth=2pt,
    rightline=false, topline=false, bottomline=false,
    linecolor=ForestGreen!70!black, backgroundcolor=ForestGreen!3,
    nobreak=false
  }
]{thmgreenbox}

\declaretheorem[style=thmgreenbox, numbered=no, name=Correct Ans]{cor}

\declaretheoremstyle[
  headfont=\bfseries\sffamily\color{black}, bodyfont=\normalfont,
  mdframed={
    linewidth=2pt,
    linecolor=black,
    nobreak=false
  }
]{thmblackline}

\declaretheorem[style=thmblackline, numbered=no, name=Description]{problem}


\usepackage{titlesec}

\titleformat{\section}
  {\normalfont\Large\bfseries}  % 字型樣式
  {}                            % 不顯示編號
  {0pt}                         % 編號和標題距離
  {Problem\ }                  % 每個 section 都會以 "Problem:" 開頭

\newcommand{\red}[1]{\textcolor{red}{#1}}
\newcommand{\blue}[1]{\textcolor{blue!70!black}{#1}}
\newcommand{\yel}[1]{\textcolor{Goldenrod!70!black}{#1}}
\newcommand{\grn}[1]{\textcolor{ForestGreen!70!black}{#1}}

\newcommand{\yelbox}[1]{%
  {\setlength{\fboxrule}{1pt}%
   \setlength{\fboxsep}{2pt}%
   \color{Goldenrod!90!black}\fbox{\normalcolor #1}}}

\newcommand{\bluebox}[1]{%
  {\setlength{\fboxrule}{1pt}%
   \setlength{\fboxsep}{2pt}%
   \color{cyan!80!black}\fbox{\normalcolor #1}}}

\newcommand{\redbox}[1]{%
  {\setlength{\fboxrule}{1pt}%
   \setlength{\fboxsep}{2pt}%
   \color{red}\fbox{\normalcolor #1}}}

\usepackage{pgfplots}
\pgfplotsset{compat=1.18}


\usepackage{adjustbox}
\usepackage{centernot}

\title{MATH-4006 Calculus 1 class 14 , Homework 1 常見錯誤}
\author{B13902126 胡允升}
\date{}  
\setlength{\parindent}{0pt}


\begin{document}
\fontsize{12pt}{16pt}\selectfont

\maketitle  
\noindent\rule{\linewidth}{0.4pt}
\textbf{For the writing style}:

\noindent\rule{\linewidth}{0.4pt}
拜託各位寫題目的時候
\begin{itemize}
    \item 如果是用電子檔寫題目可以新開一頁,把答案寫在下一頁
    \item 如果是用紙本,可以拿一張新的紙把題目標清楚
\end{itemize}
以減少助教眼壓

\noindent\rule{\linewidth}{0.4pt}

\section{1-(b)-(ii)}

\begin{problem}
Solve the following equations for x.
\[
\ln x + \ln(x - 1) = 1
\]
\end{problem}
\vspace{-12pt}
\begin{wrong}
\[
x = \frac{1 \pm \sqrt{1 + 4e}}{2}
\]
\end{wrong}
\vspace{-12pt}
\begin{cor}
\[
x = \frac{1 + \sqrt{1 + 4e}}{2}
\]
\end{cor}

\begin{reason}
For $\ln(x-1)$ to be defined, we need $x-1 > 0$, or $x > 1$.

\begin{note}
The domain of $\ln(x)$ is $(0,\infty)$
\end{note}
\vspace{0.5em}


Therefore, \[
 x = \frac{1 - \sqrt{1 + 4e}}{2}
\]
is rejected.

\end{reason}

\newpage

\section{2-(a)}

\begin{problem}
Simplify the following expressions.
\[
\sin^{-1}\left(\sin\frac{5\pi}{4}\right)
\]
\end{problem}
\vspace{-12pt}
\begin{wrong}
\[
\frac{5\pi}{4}, \frac{7\pi}{4}
\]
\end{wrong}
\vspace{-12pt}
\begin{cor}
\[
    -\frac{\pi}{4}
\]
\end{cor}

\begin{reason}
	Follow the definition from the textbook the definition $\sin^{-1}$ is 
	\vspace{0.5em}
	\begin{note}
		In page 62 of the textbook
		\[
			\sin^{-1}x = y \ \Longleftrightarrow \ \sin y = x, \ \text{ and } \ \red{-\frac{\pi}{2} \leq y \leq \frac{\pi}{2}}
		\]
	\end{note}
	\vspace{0.5em}

	The range of $\sin^{-1}$ can only be in $[-\frac{\pi}{2}, \frac{\pi}{2}]$
\end{reason}

\newpage

\section{2-(c)}
\begin{problem}
	Simplify the following expressions.
	\[
		\sin(\sec^{-1}(x)) \text{ for } x \leq -1
	\]
\end{problem}
\vspace{-12pt}
\begin{wrong}
\[
	-\frac{\sqrt{x^2-1}}{x}
\]
\end{wrong}
\vspace{-12pt}
\begin{cor}
\[
	\frac{\sqrt{x^2-1}}{x}
\]
\end{cor}

\begin{reason}
	By the Pythagoras Theorem, we can know that the answer might be
	\[
		\pm\frac{\sqrt{x^2-1}}{x}
	\]
	To get the "sign" of answer, we can first check the definition of $\sec^{-1}$ on the textbook 
	\vspace{0.01em}
	\begin{note}
		In page 64 of the textbook
		\[
			y = \sec^{-1}x \ (|x| \geq 1) \ \Longleftrightarrow \ \sec y = x, \ \text{ and } \ \red{y \in (0, \pi/2] \cup [\pi, 3\pi/2)}
		\]
	\end{note}
	\vspace{1em}
	When $x \leq -1$, $\sec^{-1}$ is between \( [\pi, 3\pi/2) \). Thus, $\sec^{-1}$ lies in the third quadrant (第三象限), which will let $\sin$ function be negative.
	Hence, the final answer is \[
		\frac{\sqrt{x^2-1}}{x}
	\]
	which is negative when $x \leq -1$.
\end{reason}

\[
\Huge \blue{\textbf{請務必對照課本,別忘記有課本}}
\]

\newpage

\section{5-(b)}
\begin{problem}
	Let \[
		g(x) = \frac{x^2-5x+6}{|x-2|}
	\]
	Does the limit $\displaystyle \lim_{x\rightarrow2} g(x)$ exist ? Explain.
\end{problem}
\vspace{-12pt}
\begin{wrong}
	Since $g(2)$ does not exist, the limit $\displaystyle \lim_{x\rightarrow2} g(x)$ does not exist.
\end{wrong}
\vspace{-12pt}
\begin{cor}
	Since $\displaystyle \lim_{x\rightarrow2^+}g(x) \neq \lim_{x\rightarrow2^-}g(x)$, the limit $\displaystyle \lim_{x\rightarrow2} g(x)$ does not exist.
\end{cor}

\begin{reason}
A function being discontinuous at a point, or even undefined there, does not necessarily mean that the limit at that point does not exist. Here is the counter example.
\[
f(x) = \frac{x^2 - 1}{x - 1}
\]
In this situation, $f(1)$ does not exist but 
\[
\lim_{x\rightarrow1^+}f(x) = \lim_{x\rightarrow1^-}f(x) = \lim_{x\rightarrow1}f(x)
\]
the limit $\displaystyle \lim_{x\rightarrow1}f(x)$ exists.
\end{reason}

\vspace{2em}

\begin{center}
	\begin{tikzpicture}
	  \begin{axis}[
		  axis lines = middle,
		  ymin=-1, ymax=4
	  ]
		% 函數 (x != 1)
		\addplot[red, thick, domain=-2:0.99] {(x^2 - 1)/(x - 1)};
		\addplot[red, thick, domain=1.01:4] {(x^2 - 1)/(x - 1)};
	
		% 缺點 (x=1, y=2) 畫空心圓
		\addplot[mark=o, mark size=3pt, only marks] coordinates {(1,2)};
	  \end{axis}
	\end{tikzpicture}
	\begin{figure}[H]
		\caption{Graph of $f$}
	\end{figure}
\end{center}

\end{document}