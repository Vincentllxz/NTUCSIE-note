\documentclass{article}
% basics
\usepackage{xeCJK}
\setCJKmainfont{Kaiti TC} % 新細明體
\setCJKmainfont{Kaiti TC}[AutoFakeBold=6 , AutoFakeSlant=.2]

\usepackage[letterpaper,top=2cm,bottom=2cm,left=3cm,right=3cm]{geometry}
\linespread{1.2}
\usepackage{amssymb}
\usepackage{amsmath}
\usepackage{graphicx}
\usepackage{amsthm}
\usepackage[dvipsnames]{xcolor}
\usepackage{xeCJK} % 啟用中文支持
\usepackage{circuitikz} % 電路圖
\usepackage{tcolorbox} % 用來加方框
\usepackage{algorithm}
\usepackage[noend]{algpseudocode}
\usepackage{enumitem}
\usepackage{float}
\usepackage{tikz}
\usetikzlibrary{trees}
\usepackage{hyperref}
\usetikzlibrary{arrows.meta,automata,positioning}

% === theorem environments with mdframed ===
\usepackage{thmtools}   % 提供 declaretheorem
\usepackage{mdframed}   % theorem style 加邊框

\declaretheoremstyle[
  headfont=\bfseries\sffamily\color{black}, bodyfont=\normalfont,
  mdframed={
    linewidth=2pt,
    rightline=false, topline=false, bottomline=false,
    linecolor=black,
    nobreak=false
  }
]{thmblueline}

\declaretheorem[style=thmblueline, numbered=no, name=Reason]{reason}

\declaretheoremstyle[
  headfont=\bfseries\sffamily\color{black}, bodyfont=\normalfont,
  mdframed={
    linewidth=2pt,
    rightline=false, topline=false, bottomline=false,
    linecolor=NavyBlue!60!black,
    nobreak=false
  }
]{thmgreenline}

\declaretheorem[style=thmgreenline, numbered=no, name=Note]{note}

\declaretheoremstyle[
  headfont=\bfseries\sffamily\color{red!70!black}, bodyfont=\normalfont,
  mdframed={
    linewidth=2pt,
    rightline=false, topline=false, bottomline=false,
    linecolor=red, backgroundcolor=red!3,
    nobreak=false
  }
]{thmredbox}

\declaretheorem[style=thmredbox, numbered=no, name=Wrong Ans]{wrong}


\declaretheoremstyle[
  headfont=\bfseries\sffamily\color{ForestGreen!70!black}, bodyfont=\normalfont,
  mdframed={
    linewidth=2pt,
    rightline=false, topline=false, bottomline=false,
    linecolor=ForestGreen!70!black, backgroundcolor=ForestGreen!3,
    nobreak=false
  }
]{thmgreenbox}

\declaretheorem[style=thmgreenbox, numbered=no, name=Correct Ans]{cor}

\declaretheoremstyle[
  headfont=\bfseries\sffamily\color{black}, bodyfont=\normalfont,
  mdframed={
    linewidth=2pt,
    linecolor=black,
    nobreak=false
  }
]{thmblackline}

\declaretheorem[style=thmblackline, numbered=no, name=Description]{problem}


\usepackage{titlesec}

\titleformat{\section}
  {\normalfont\Large\bfseries}  % 字型樣式
  {}                            % 不顯示編號
  {0pt}                         % 編號和標題距離
  {Problem\ }                  % 每個 section 都會以 "Problem:" 開頭

\newcommand{\red}[1]{\textcolor{red}{#1}}
\newcommand{\blue}[1]{\textcolor{blue!70!black}{#1}}
\newcommand{\yel}[1]{\textcolor{Goldenrod!70!black}{#1}}
\newcommand{\grn}[1]{\textcolor{ForestGreen!70!black}{#1}}

\newcommand{\yelbox}[1]{%
  {\setlength{\fboxrule}{1pt}%
   \setlength{\fboxsep}{2pt}%
   \color{Goldenrod!90!black}\fbox{\normalcolor #1}}}

\newcommand{\bluebox}[1]{%
  {\setlength{\fboxrule}{1pt}%
   \setlength{\fboxsep}{2pt}%
   \color{cyan!80!black}\fbox{\normalcolor #1}}}

\newcommand{\redbox}[1]{%
  {\setlength{\fboxrule}{1pt}%
   \setlength{\fboxsep}{2pt}%
   \color{red}\fbox{\normalcolor #1}}}

\usepackage{pgfplots}
\pgfplotsset{compat=1.18}


\usepackage{adjustbox}
\usepackage{centernot}

\title{MATH-4007 Calculus 2 class 14 , Homework 11 常見錯誤}
\author{B13902126 資工二 胡允升}
\date{}  
\setlength{\parindent}{0pt}


\begin{document}
\fontsize{12pt}{16pt}\selectfont

\maketitle  
\noindent\rule{\linewidth}{0.4pt}
\textbf{For the writing style}:

\noindent\rule{\linewidth}{0.4pt}
\textbf{再再再再再再次提醒拜託各位寫題目的時候}
\begin{itemize}
    \item 如果是用電子檔寫題目可以「新開一頁」,把答案寫在下一頁
    \item 如果是用紙本,可以拿一張新的紙把題目標清楚
    \item 標示清楚計算過程
\end{itemize}
以減少助教眼壓。

\noindent\rule{\linewidth}{0.4pt}

\vspace{2em}

\noindent\rule{\linewidth}{0.4pt}
\textbf{For studying(關於課程網上的影片)}:

\noindent\rule{\linewidth}{0.4pt}

教授上課不一定能夠 cover 到所有題目,助教們在 NTUCOOL 上都會放上「詳解」影片,\textbf{請務必要觀看},在這次題目中,有不少題都是教授上課沒有 cover 到的講義內容變化題,明顯的很多人並沒有觀看助教們錄好的影片,希望大家能夠確實觀看,才能在期考拿下高分。

\noindent\rule{\linewidth}{0.4pt}

\noindent\rule{\linewidth}{0.4pt}
\textbf{For questioning}:

\noindent\rule{\linewidth}{0.4pt}

有任何問題都可以來信問助教,WebWork 題目也可以在寄信給助教的時候一併詢問。

\noindent\rule{\linewidth}{0.4pt}

\newpage

\section{2-(c)}

\begin{problem}
Find the following the indefinite integral
\[
	\int \frac{\text{d} x}{\sqrt{x(x-6)}}
\]
\end{problem}
\vspace{-12pt}
\begin{cor}
	\[
		\ln \left| \left( \frac{x-3}{3} \right) + \sqrt{\left( \frac{x-3}{3} \right)^2 - 1} \right| + C
	\]
\end{cor}

\begin{reason}
	Note that by completing the square(配方法), we have \[
		x(x-6) = (x-3)^2 - 9
	\]
	Therefore, we do the substitution: \[
		x-3 = 3 \sec \theta \ (\text{d} x = 3 \sec \theta \tan \theta \ \text{d} \theta)
	\] and hence,
	\begin{align*}
		\int \frac{\text{d} x}{\sqrt{x(x-6)}}
		&= \int \frac{3 \sec \theta \tan \theta \ \text{d} \theta}{\sqrt{(3 \sec \theta)^2 - 9}} \\
		&= \int \frac{3 \sec \theta \tan \theta \ \text{d} \theta}{3 \sqrt{\sec^2 \theta - 1}} \\	
		&= \int \sec \theta \ \text{d} \theta \\
		&= \ln | \sec \theta + \tan \theta | + C \\
		&= \ln \left| \left( \frac{x-3}{3} \right) + \sqrt{\left( \frac{x-3}{3} \right)^2 - 1} \right| + C
	\end{align*}\hfill \qed

	\begin{note}
		We can reduce the expression a bit more: \begin{align*}
		\ln \left| \left( \frac{x-3}{3} \right) + \sqrt{\left( \frac{x-3}{3} \right)^2 - 1} \right| + C
		&= \ln \left| \frac{x-3 + \sqrt{(x-3)^2 - 9}}{3} \right| + C \\
		&= \ln \left| x-3 + \sqrt{x(x-6)} \right| - \ln 3 + C \\
		&= \ln \left| x-3 + \sqrt{x(x-6)} \right| + C'
		\end{align*} \hfill \qed
	\end{note}
\end{reason}

\newpage

\section{4-(a)}

\begin{problem}
	Let $\displaystyle I_n = \int sec^n(x) \ \text{d} x$. Prove that \[
		I_n = \frac{1}{n - 1} \tan x \sec^{n - 2} (x) + \frac{n - 2}{n - 1} I_{n - 2}
	\]
\end{problem}

\begin{reason}
	First imply the integration by parts:
	\[
		\begin{tabular}{c|cc}
			 & D & I \\
			\hline
			+ & $\sec^{n - 2} (x)$ & $\sec x \tan x$ \\
			- & $(n - 2) \sec^{n - 3} (x) \sec x \tan x$ & $\sec x$ \\
		\end{tabular}
	\]
	Then we get \begin{align*}
	I_n 
	&= \int \sec^{\,n-2}(x)\,\sec^{2}(x)\,dx \\[6pt]
	&= \tan(x)\sec^{\,n-2}(x) 
		- (n-2)\int \sec^{\,n-2}(x)\tan^{2}(x)\,dx \\[6pt]
	&= \tan(x)\sec^{\,n-2}(x) 
		- (n-2)\int \sec^{\,n-2}(x)(\sec^{2}x - 1)\,dx \\[6pt]
	&= \tan(x)\sec^{\,n-2}(x)
		- (n-2)\int \underset{I_n}{\underbrace{\sec^{n}(x)}}\,dx
		+ (n-2)\int \underset{I_{n-2}}{\underbrace{\sec^{\,n-2}(x)}}\,dx \\[6pt]
	&= \tan(x)\sec^{\,n-2}(x)
		- (n-2)I_n
		+ (n-2)I_{n-2}
	\end{align*}

	Rearranging the equation, we have \[
		I_n = \frac{1}{n - 1} \tan x \sec^{n - 2} (x) + \frac{n - 2}{n - 1} I_{n - 2}
	\]
	\hfill \qed
\end{reason}

\newpage

\section{WebWork}

\begin{problem}
Find the following the indefinite integral
\[
	\int \frac{5}{[(ax)^2 + b^2]^{3/2}} \text{d} x
\]
\end{problem}
\vspace{-12pt}
\begin{cor}
	\[ 
		-\frac{5}{a b \sqrt{b^2 - a^2 x^2}} + C
	\]
\end{cor}

\begin{reason}
	We have to convert it into a standard form \[
		\sec^2 \theta = 1 + \tan^2 \theta
	\]
	So we let $\displaystyle x = \frac{b}{a} \sec \theta$
	and we get \begin{align*}
		\int \frac{5}{[(ax)^2 + b^2]^{3/2}} \text{d} x &= \int \frac{5}{[a^2 (\frac{b}{a} \sec \theta)^2 + b^2]^{3/2}} \cdot \frac{b}{a} \sec \theta \tan \theta \ \text{d} \theta \\
		&= \int \frac{5}{[b^2 \sec^2 \theta + b^2]^{3/2}} \cdot \frac{b}{a} \sec \theta \tan \theta \ \text{d} \theta \\
		&= \int \frac{5}{b^3 (\sec^2 \theta)^{3/2}} \cdot \frac{b}{a} \sec \theta \tan \theta \ \text{d} \theta \\
		&= \frac{5}{a b^2} \int \frac{\cos \theta}{\sin^2 \theta} \ \text{d} \theta
	\end{align*}
	then, let $u = \sin \theta \ (\text{d} \theta = \frac{1}{\cos \theta} \text{d} u)$, we have \begin{align*}
		\frac{5}{a b^2} \int \frac{\cos \theta}{\sin^2 \theta} \ \text{d} \theta = \frac{5}{a b^2} \int \frac{\cos \theta}{u^2} \cdot \frac{1}{\cos \theta} \text{d} u
		&= \frac{5}{a b^2} \int u^{-2} \text{d} u \\
		&= \frac{5}{a b^2} \cdot (-u^{-1}) + C \\
		&= -\frac{5}{a b^2 \sin \theta} + C
	\end{align*}
	Finally, we have to convert back to $x$: \[
		\sin \theta = \sqrt{1 - \cos^2 \theta} = \sqrt{1 - \left( \frac{a x}{b} \right)^2} = \frac{\sqrt{b^2 - a^2 x^2}}{b}
	\]
	Thus, the final answer is \[
		-\frac{5}{a b^2} \cdot \frac{b}{\sqrt{b^2 - a^2 x^2}} + C = -\frac{5}{a b \sqrt{b^2 - a^2 x^2}} + C
	\]
	\hfill \qed
\end{reason}

\end{document}