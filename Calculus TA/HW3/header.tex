% basics
\usepackage{xeCJK}
\setCJKmainfont{Kaiti TC} % 新細明體
\setCJKmainfont{Kaiti TC}[AutoFakeBold=6 , AutoFakeSlant=.2]

\usepackage[letterpaper,top=2cm,bottom=2cm,left=3cm,right=3cm]{geometry}
\linespread{1.2}
\usepackage{amssymb}
\usepackage{amsmath}
\usepackage{graphicx}
\usepackage{amsthm}
\usepackage[dvipsnames]{xcolor}
\usepackage{xeCJK} % 啟用中文支持
\usepackage{circuitikz} % 電路圖
\usepackage{tcolorbox} % 用來加方框
\usepackage{algorithm}
\usepackage[noend]{algpseudocode}
\usepackage{enumitem}
\usepackage{float}
\usepackage{tikz}
\usetikzlibrary{trees}
\usepackage{hyperref}
\usetikzlibrary{arrows.meta,automata,positioning}

% === theorem environments with mdframed ===
\usepackage{thmtools}   % 提供 declaretheorem
\usepackage{mdframed}   % theorem style 加邊框

\declaretheoremstyle[
  headfont=\bfseries\sffamily\color{black}, bodyfont=\normalfont,
  mdframed={
    linewidth=2pt,
    rightline=false, topline=false, bottomline=false,
    linecolor=black,
    nobreak=false
  }
]{thmblueline}

\declaretheorem[style=thmblueline, numbered=no, name=Reason]{reason}

\declaretheoremstyle[
  headfont=\bfseries\sffamily\color{black}, bodyfont=\normalfont,
  mdframed={
    linewidth=2pt,
    rightline=false, topline=false, bottomline=false,
    linecolor=NavyBlue!60!black,
    nobreak=false
  }
]{thmgreenline}

\declaretheorem[style=thmgreenline, numbered=no, name=Note]{note}

\declaretheoremstyle[
  headfont=\bfseries\sffamily\color{red!70!black}, bodyfont=\normalfont,
  mdframed={
    linewidth=2pt,
    rightline=false, topline=false, bottomline=false,
    linecolor=red, backgroundcolor=red!3,
    nobreak=false
  }
]{thmredbox}

\declaretheorem[style=thmredbox, numbered=no, name=Wrong Ans]{wrong}


\declaretheoremstyle[
  headfont=\bfseries\sffamily\color{ForestGreen!70!black}, bodyfont=\normalfont,
  mdframed={
    linewidth=2pt,
    rightline=false, topline=false, bottomline=false,
    linecolor=ForestGreen!70!black, backgroundcolor=ForestGreen!3,
    nobreak=false
  }
]{thmgreenbox}

\declaretheorem[style=thmgreenbox, numbered=no, name=Correct Ans]{cor}

\declaretheoremstyle[
  headfont=\bfseries\sffamily\color{black}, bodyfont=\normalfont,
  mdframed={
    linewidth=2pt,
    linecolor=black,
    nobreak=false
  }
]{thmblackline}

\declaretheorem[style=thmblackline, numbered=no, name=Description]{problem}


\usepackage{titlesec}

\titleformat{\section}
  {\normalfont\Large\bfseries}  % 字型樣式
  {}                            % 不顯示編號
  {0pt}                         % 編號和標題距離
  {Problem\ }                  % 每個 section 都會以 "Problem:" 開頭

\newcommand{\red}[1]{\textcolor{red}{#1}}
\newcommand{\blue}[1]{\textcolor{blue!70!black}{#1}}
\newcommand{\yel}[1]{\textcolor{Goldenrod!70!black}{#1}}
\newcommand{\grn}[1]{\textcolor{ForestGreen!70!black}{#1}}

\newcommand{\yelbox}[1]{%
  {\setlength{\fboxrule}{1pt}%
   \setlength{\fboxsep}{2pt}%
   \color{Goldenrod!90!black}\fbox{\normalcolor #1}}}

\newcommand{\bluebox}[1]{%
  {\setlength{\fboxrule}{1pt}%
   \setlength{\fboxsep}{2pt}%
   \color{cyan!80!black}\fbox{\normalcolor #1}}}

\newcommand{\redbox}[1]{%
  {\setlength{\fboxrule}{1pt}%
   \setlength{\fboxsep}{2pt}%
   \color{red}\fbox{\normalcolor #1}}}

\usepackage{pgfplots}
\pgfplotsset{compat=1.18}
