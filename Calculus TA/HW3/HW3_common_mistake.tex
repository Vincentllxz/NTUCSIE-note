\documentclass{article}
% basics
\usepackage{xeCJK}
\setCJKmainfont{Kaiti TC} % 新細明體
\setCJKmainfont{Kaiti TC}[AutoFakeBold=6 , AutoFakeSlant=.2]

\usepackage[letterpaper,top=2cm,bottom=2cm,left=3cm,right=3cm]{geometry}
\linespread{1.2}
\usepackage{amssymb}
\usepackage{amsmath}
\usepackage{graphicx}
\usepackage{amsthm}
\usepackage[dvipsnames]{xcolor}
\usepackage{xeCJK} % 啟用中文支持
\usepackage{circuitikz} % 電路圖
\usepackage{tcolorbox} % 用來加方框
\usepackage{algorithm}
\usepackage[noend]{algpseudocode}
\usepackage{enumitem}
\usepackage{float}
\usepackage{tikz}
\usetikzlibrary{trees}
\usepackage{hyperref}
\usetikzlibrary{arrows.meta,automata,positioning}

% === theorem environments with mdframed ===
\usepackage{thmtools}   % 提供 declaretheorem
\usepackage{mdframed}   % theorem style 加邊框

\declaretheoremstyle[
  headfont=\bfseries\sffamily\color{black}, bodyfont=\normalfont,
  mdframed={
    linewidth=2pt,
    rightline=false, topline=false, bottomline=false,
    linecolor=black,
    nobreak=false
  }
]{thmblueline}

\declaretheorem[style=thmblueline, numbered=no, name=Reason]{reason}

\declaretheoremstyle[
  headfont=\bfseries\sffamily\color{black}, bodyfont=\normalfont,
  mdframed={
    linewidth=2pt,
    rightline=false, topline=false, bottomline=false,
    linecolor=NavyBlue!60!black,
    nobreak=false
  }
]{thmgreenline}

\declaretheorem[style=thmgreenline, numbered=no, name=Note]{note}

\declaretheoremstyle[
  headfont=\bfseries\sffamily\color{red!70!black}, bodyfont=\normalfont,
  mdframed={
    linewidth=2pt,
    rightline=false, topline=false, bottomline=false,
    linecolor=red, backgroundcolor=red!3,
    nobreak=false
  }
]{thmredbox}

\declaretheorem[style=thmredbox, numbered=no, name=Wrong Ans]{wrong}


\declaretheoremstyle[
  headfont=\bfseries\sffamily\color{ForestGreen!70!black}, bodyfont=\normalfont,
  mdframed={
    linewidth=2pt,
    rightline=false, topline=false, bottomline=false,
    linecolor=ForestGreen!70!black, backgroundcolor=ForestGreen!3,
    nobreak=false
  }
]{thmgreenbox}

\declaretheorem[style=thmgreenbox, numbered=no, name=Correct Ans]{cor}

\declaretheoremstyle[
  headfont=\bfseries\sffamily\color{black}, bodyfont=\normalfont,
  mdframed={
    linewidth=2pt,
    linecolor=black,
    nobreak=false
  }
]{thmblackline}

\declaretheorem[style=thmblackline, numbered=no, name=Description]{problem}


\usepackage{titlesec}

\titleformat{\section}
  {\normalfont\Large\bfseries}  % 字型樣式
  {}                            % 不顯示編號
  {0pt}                         % 編號和標題距離
  {Problem\ }                  % 每個 section 都會以 "Problem:" 開頭

\newcommand{\red}[1]{\textcolor{red}{#1}}
\newcommand{\blue}[1]{\textcolor{blue!70!black}{#1}}
\newcommand{\yel}[1]{\textcolor{Goldenrod!70!black}{#1}}
\newcommand{\grn}[1]{\textcolor{ForestGreen!70!black}{#1}}

\newcommand{\yelbox}[1]{%
  {\setlength{\fboxrule}{1pt}%
   \setlength{\fboxsep}{2pt}%
   \color{Goldenrod!90!black}\fbox{\normalcolor #1}}}

\newcommand{\bluebox}[1]{%
  {\setlength{\fboxrule}{1pt}%
   \setlength{\fboxsep}{2pt}%
   \color{cyan!80!black}\fbox{\normalcolor #1}}}

\newcommand{\redbox}[1]{%
  {\setlength{\fboxrule}{1pt}%
   \setlength{\fboxsep}{2pt}%
   \color{red}\fbox{\normalcolor #1}}}

\usepackage{pgfplots}
\pgfplotsset{compat=1.18}


\usepackage{adjustbox}
\usepackage{centernot}

\title{MATH-4006 Calculus 1 class 14 , Homework 3 常見錯誤}
\author{B13902126 胡允升}
\date{}  
\setlength{\parindent}{0pt}


\begin{document}
\fontsize{12pt}{16pt}\selectfont

\maketitle  
\noindent\rule{\linewidth}{0.4pt}
\textbf{For the writing style}:

\noindent\rule{\linewidth}{0.4pt}
再次提醒拜託各位寫題目的時候
\begin{itemize}
    \item 如果是用電子檔寫題目可以「新開一頁」,把答案寫在下一頁
    \item 如果是用紙本,可以拿一張新的紙把題目標清楚
\end{itemize}
以減少助教眼壓

\noindent\rule{\linewidth}{0.4pt}

\section{1-(b)}

\begin{problem}
Consider the fucntion
\[
Q(x) = \lim_{h \to 0} \frac{1 + x + x^2 + x e^x}{1 - x + x^2 - x e^x} 
\]
Evaluate $Q'(0).$ \\
(\texttt{Hint}: it's easier to compute this by the definition of derivative.)
\end{problem}
\vspace{-12pt}
\begin{wrong}
	直接微分
\end{wrong}
\vspace{-12pt}
\begin{cor}
用定義作答
\end{cor}

\begin{reason}
題目有請大家使用微分的定義作答,請不要直接把它微分

\[
Q'(0) 
= \lim_{h \to 0} \frac{Q(h) - Q(0)}{h} 
= \lim_{h \to 0} \frac{\frac{1 + h + h^2 + h e^h}{1 - h + h^2 - h e^h} - 1}{h} 
= \lim_{h \to 0} \frac{2 + 2e^h}{1 - h + h^2 - h e^h} 
= 4.
\]

\end{reason}

\newpage

\section{2-(a)}

\begin{problem}
Prove that
\[
	f(x) = \sin(x^{1/3})
\]
is not differentiable at $x=0$.
\end{problem}
\vspace{-12pt}
\begin{wrong}[1]
\[
\lim_{h \to 0} \frac{f(h) - f(0)}{h} \quad \text{Does Not Exists}
\]
\end{wrong}
\vspace{-12pt}
\begin{wrong}[2]
\[
\lim_{h \to 0} f'(h) \quad \text{Does Not Exists}
\]
\end{wrong}
\vspace{-12pt}
\begin{cor}
\[
\lim_{h \to 0} \frac{f(h) - f(0)}{h} = +\infty
\]
, hence, $f(x)$ is not differentiable at $x=0$.
\end{cor}

\begin{reason}
	As
	\[
	\lim_{h \to 0} \frac{f(h) - f(0)}{h} 
	= \lim_{h \to 0} \frac{\sin(h^{1/3}) - 0}{h} 
	= \lim_{h \to 0} \frac{\sin(h^{1/3})}{h^{1/3}} \cdot \frac{1}{h^{2/3}} 
	= +\infty.
	\]
	, hence $f(x)$ is not differentiable at $x=0$.
	\vspace{0.5em}
	\begin{note}
	\red{Wrong (2)} is incorrect. 因為這個陳述只證明了 $f'(x)$ 在 $x=0$ 是不連續的,並沒有證明他的 differentiability
	\end{note}
	\vspace{0.5em}

\end{reason}

\newpage

\section{5}
\begin{problem}
Consider the following function:
\[
f(x) =
\begin{cases}
x^2, & \text{if } x \le 2, \\[6pt]
mx + b, & \text{if } x > 2.
\end{cases}
\]

It is known that \( f \) is differentiable everywhere. Find the values of \( m \) and \( b \).

\end{problem}
\vspace{-12pt}
\begin{cor}
	$m=4, b=-4$
\end{cor}

\begin{reason}
	因為可微分性蘊含連續性,所以函數 $f$ 在 $x = 2$ 處是連續的。
	這意味著依定義有
	\[
	\lim_{x \to 2^+} f(x) = f(2),
	\]
	因此我們得到 $2m + b = 4$。

	接著,$f$ 在 $x = 2$ 可微分,這等價於極限
	\[
	\lim_{x \to 2} \frac{f(x) - f(2)}{x - 2}
	\]
	存在,這蘊含了
	\[
	\lim_{x \to 2^-} \frac{f(x) - f(2)}{x - 2}
	= \lim_{x \to 2^+} \frac{f(x) - f(2)}{x - 2}.
	\]

	因此我們得到
	\[
	\lim_{x \to 2^-} \frac{x^2 - 4}{x - 2}
	= \lim_{x \to 2^+} \frac{mx + b - 4}{x - 2}
	= \lim_{x \to 2^+} \frac{mx - 2m}{x - 2}.
	\]

	最後一個等式來自於我們已知 $2m + b = 4$。因此 $m = 4$。
	將其代入 $2m + b = 4$,得到 $b = -4$。
	\begin{note}
	將
	\[
	\lim_{x \to 2^+} f'(x) = \lim_{x \to 2^-} f'(x)
	\]
	這樣設定是不正確的 —— 雖然你可能會得到相同的答案,
	但那只是因為「剛好」給定的 $f'(x)$ 是連續的;
	在一般情況下並不一定如此。
	\end{note}

\end{reason}

\end{document}