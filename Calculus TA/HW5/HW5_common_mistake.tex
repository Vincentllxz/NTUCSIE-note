\documentclass{article}
% basics
\usepackage{xeCJK}
\setCJKmainfont{Kaiti TC} % 新細明體
\setCJKmainfont{Kaiti TC}[AutoFakeBold=6 , AutoFakeSlant=.2]

\usepackage[letterpaper,top=2cm,bottom=2cm,left=3cm,right=3cm]{geometry}
\linespread{1.2}
\usepackage{amssymb}
\usepackage{amsmath}
\usepackage{graphicx}
\usepackage{amsthm}
\usepackage[dvipsnames]{xcolor}
\usepackage{xeCJK} % 啟用中文支持
\usepackage{circuitikz} % 電路圖
\usepackage{tcolorbox} % 用來加方框
\usepackage{algorithm}
\usepackage[noend]{algpseudocode}
\usepackage{enumitem}
\usepackage{float}
\usepackage{tikz}
\usetikzlibrary{trees}
\usepackage{hyperref}
\usetikzlibrary{arrows.meta,automata,positioning}

% === theorem environments with mdframed ===
\usepackage{thmtools}   % 提供 declaretheorem
\usepackage{mdframed}   % theorem style 加邊框

\declaretheoremstyle[
  headfont=\bfseries\sffamily\color{black}, bodyfont=\normalfont,
  mdframed={
    linewidth=2pt,
    rightline=false, topline=false, bottomline=false,
    linecolor=black,
    nobreak=false
  }
]{thmblueline}

\declaretheorem[style=thmblueline, numbered=no, name=Reason]{reason}

\declaretheoremstyle[
  headfont=\bfseries\sffamily\color{black}, bodyfont=\normalfont,
  mdframed={
    linewidth=2pt,
    rightline=false, topline=false, bottomline=false,
    linecolor=NavyBlue!60!black,
    nobreak=false
  }
]{thmgreenline}

\declaretheorem[style=thmgreenline, numbered=no, name=Note]{note}

\declaretheoremstyle[
  headfont=\bfseries\sffamily\color{red!70!black}, bodyfont=\normalfont,
  mdframed={
    linewidth=2pt,
    rightline=false, topline=false, bottomline=false,
    linecolor=red, backgroundcolor=red!3,
    nobreak=false
  }
]{thmredbox}

\declaretheorem[style=thmredbox, numbered=no, name=Wrong Ans]{wrong}


\declaretheoremstyle[
  headfont=\bfseries\sffamily\color{ForestGreen!70!black}, bodyfont=\normalfont,
  mdframed={
    linewidth=2pt,
    rightline=false, topline=false, bottomline=false,
    linecolor=ForestGreen!70!black, backgroundcolor=ForestGreen!3,
    nobreak=false
  }
]{thmgreenbox}

\declaretheorem[style=thmgreenbox, numbered=no, name=Correct Ans]{cor}

\declaretheoremstyle[
  headfont=\bfseries\sffamily\color{black}, bodyfont=\normalfont,
  mdframed={
    linewidth=2pt,
    linecolor=black,
    nobreak=false
  }
]{thmblackline}

\declaretheorem[style=thmblackline, numbered=no, name=Description]{problem}


\usepackage{titlesec}

\titleformat{\section}
  {\normalfont\Large\bfseries}  % 字型樣式
  {}                            % 不顯示編號
  {0pt}                         % 編號和標題距離
  {Problem\ }                  % 每個 section 都會以 "Problem:" 開頭

\newcommand{\red}[1]{\textcolor{red}{#1}}
\newcommand{\blue}[1]{\textcolor{blue!70!black}{#1}}
\newcommand{\yel}[1]{\textcolor{Goldenrod!70!black}{#1}}
\newcommand{\grn}[1]{\textcolor{ForestGreen!70!black}{#1}}

\newcommand{\yelbox}[1]{%
  {\setlength{\fboxrule}{1pt}%
   \setlength{\fboxsep}{2pt}%
   \color{Goldenrod!90!black}\fbox{\normalcolor #1}}}

\newcommand{\bluebox}[1]{%
  {\setlength{\fboxrule}{1pt}%
   \setlength{\fboxsep}{2pt}%
   \color{cyan!80!black}\fbox{\normalcolor #1}}}

\newcommand{\redbox}[1]{%
  {\setlength{\fboxrule}{1pt}%
   \setlength{\fboxsep}{2pt}%
   \color{red}\fbox{\normalcolor #1}}}

\usepackage{pgfplots}
\pgfplotsset{compat=1.18}


\usepackage{adjustbox}
\usepackage{centernot}

\title{MATH-4006 Calculus 1 class 14 , Homework 5 常見錯誤}
\author{B13902126 胡允升}
\date{}  
\setlength{\parindent}{0pt}


\begin{document}
\fontsize{12pt}{16pt}\selectfont

\maketitle  
\noindent\rule{\linewidth}{0.4pt}
\textbf{For the writing style}:

\noindent\rule{\linewidth}{0.4pt}
\textbf{再再再再再再次提醒拜託各位寫題目的時候}
\begin{itemize}
    \item 如果是用電子檔寫題目可以「新開一頁」,把答案寫在下一頁
    \item 如果是用紙本,可以拿一張新的紙把題目標清楚
    \item 標示清楚計算過程
\end{itemize}
以減少助教眼壓。

\noindent\rule{\linewidth}{0.4pt}

\vspace{2em}

\noindent\rule{\linewidth}{0.4pt}
\textbf{For studying(關於課程網上的影片)}:

\noindent\rule{\linewidth}{0.4pt}

教授上課不一定能夠 cover 到所有題目,助教們在 NTUCOOL 上都會放上「詳解」影片,\textbf{請務必要觀看},在這次題目中,有不少題都是教授上課沒有 cover 到的講義內容變化題,明顯的很多人並沒有觀看助教們錄好的影片,希望大家能夠確實觀看,才能在期考拿下高分。

\noindent\rule{\linewidth}{0.4pt}


\newpage

\section{4-(b)}

\begin{problem}
For each of the following limits, firstly write down which kinds of indeterminate powers ($1^\infty$ or $0^0$ or $\infty^0$) it is. Then compute them by using the L'Hospital's Rule appropriately.
\[
\lim_{x \rightarrow \infty} \left(\frac{2}{\pi} \tan^{-1}(x) \right)^{x}
\]
\end{problem}
\vspace{-12pt}
\begin{cor}
	\[ e^{-\frac{2}{\pi}} \]
\end{cor}

\begin{reason}
	這是一個「$1^\infty$」型式。

令
\[
y = \left( \frac{2}{\pi} \tan^{-1} x \right)^x .
\]

\begin{align*}
\lim_{x \to \infty} \ln y
&= \lim_{x \to \infty} 
\frac{\ln\left( \frac{2}{\pi} \tan^{-1} x \right)}{\frac{1}{x}} \\[1ex]
&\overset{0/0}{=}
\lim_{x \to \infty}
\frac{\frac{\pi}{2 \tan^{-1}(x)} \cdot \frac{2}{\pi} \cdot \frac{1}{1+x^2}}{-\frac{1}{x^2}} \\[1ex]
&= \lim_{x \to \infty}
\frac{-x^2}{\tan^{-1}(x)(1+x^2)} \\[1ex]
&= \lim_{x \to \infty}
\frac{1}{\tan^{-1}(x)} \cdot \frac{-x^2}{1+x^2} \\[1ex]
&= \frac{1}{\pi/2} \cdot (-1) \\[1ex]
&= -\frac{2}{\pi}.
\end{align*}

由此可得
\[
\lim_{x \to \infty} \left( \frac{2}{\pi} \tan^{-1} x \right)^x
= e^{-\frac{2}{\pi}} .
\]

\end{reason}

\newpage

\section{5}

\begin{problem}
Prove that the equation $$2\tan^{-1} x + e^x = 0$$ has a unique solution in $x$.
\end{problem}
\vspace{-12pt}
\begin{wrong}
	只用嚴格遞增 (Strictly Increasing) 說明
\end{wrong}
\vspace{-12pt}
\begin{cor}
	用 IVT (Intermediate Value Theorem) 中間值定理說明他有解,再用 MVT (Mean Value Theorem) 平均值定理說明解唯一(或是 Strictly Increasing)
\end{cor}

\begin{reason}
	(英文版請見詳解,中文有點難懂)\\
	要證明解唯一,我們必須證明兩個部分:
	\begin{itemize}
		\item \textbf{存在性}:因為 $F$ 是連續函數,
		\[
		F(-1) = -\frac{\pi}{2} + \frac{1}{e} < 0, \qquad
		F(1) = \frac{\pi}{2} + e > 0,
		\]
		根據中間值定理(IVT, Intermediate Value Theorem),在 $(-1, 1)$ 之間必定存在某個 $c$ 使得 $F(c) = 0$。
		
		\item \textbf{唯一性}:假設 $\alpha$ 與 $\beta$ 是 $F(x) = 0$ 的兩個不同解。  
		根據洛爾定理(Rolle's Theorem),在 $\alpha$ 與 $\beta$ 之間必定存在某個 $d$ 使得 $F'(d) = 0$。  
		
		然而,
		\[
		F'(x) = \frac{2}{1 + x^2} + e^x \neq 0 \quad \text{對所有 } x \text{皆成立}。
		\]
		這導致矛盾,因此方程式至多只能有一個解。
	\end{itemize}

	因此,這個方程式在 $(-1, 1)$ 之間有唯一解。

	\begin{note}
		第二部分\textbf{Uniqueness} 也可以用 Strictly Increasing 說明,但其實 Strictly Increasing 導致解唯一也只是 MVT 的一個推論而已。
	\end{note}
\end{reason}

\[
\Huge \blue{\textbf{請務對照講義 p47,以及(我錄的)影片}}
\]

\end{document}