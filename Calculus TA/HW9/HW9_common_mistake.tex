\documentclass{article}
% basics
\usepackage{xeCJK}
\setCJKmainfont{Kaiti TC} % 新細明體
\setCJKmainfont{Kaiti TC}[AutoFakeBold=6 , AutoFakeSlant=.2]

\usepackage[letterpaper,top=2cm,bottom=2cm,left=3cm,right=3cm]{geometry}
\linespread{1.2}
\usepackage{amssymb}
\usepackage{amsmath}
\usepackage{graphicx}
\usepackage{amsthm}
\usepackage[dvipsnames]{xcolor}
\usepackage{xeCJK} % 啟用中文支持
\usepackage{circuitikz} % 電路圖
\usepackage{tcolorbox} % 用來加方框
\usepackage{algorithm}
\usepackage[noend]{algpseudocode}
\usepackage{enumitem}
\usepackage{float}
\usepackage{tikz}
\usetikzlibrary{trees}
\usepackage{hyperref}
\usetikzlibrary{arrows.meta,automata,positioning}

% === theorem environments with mdframed ===
\usepackage{thmtools}   % 提供 declaretheorem
\usepackage{mdframed}   % theorem style 加邊框

\declaretheoremstyle[
  headfont=\bfseries\sffamily\color{black}, bodyfont=\normalfont,
  mdframed={
    linewidth=2pt,
    rightline=false, topline=false, bottomline=false,
    linecolor=black,
    nobreak=false
  }
]{thmblueline}

\declaretheorem[style=thmblueline, numbered=no, name=Reason]{reason}

\declaretheoremstyle[
  headfont=\bfseries\sffamily\color{black}, bodyfont=\normalfont,
  mdframed={
    linewidth=2pt,
    rightline=false, topline=false, bottomline=false,
    linecolor=NavyBlue!60!black,
    nobreak=false
  }
]{thmgreenline}

\declaretheorem[style=thmgreenline, numbered=no, name=Note]{note}

\declaretheoremstyle[
  headfont=\bfseries\sffamily\color{red!70!black}, bodyfont=\normalfont,
  mdframed={
    linewidth=2pt,
    rightline=false, topline=false, bottomline=false,
    linecolor=red, backgroundcolor=red!3,
    nobreak=false
  }
]{thmredbox}

\declaretheorem[style=thmredbox, numbered=no, name=Wrong Ans]{wrong}


\declaretheoremstyle[
  headfont=\bfseries\sffamily\color{ForestGreen!70!black}, bodyfont=\normalfont,
  mdframed={
    linewidth=2pt,
    rightline=false, topline=false, bottomline=false,
    linecolor=ForestGreen!70!black, backgroundcolor=ForestGreen!3,
    nobreak=false
  }
]{thmgreenbox}

\declaretheorem[style=thmgreenbox, numbered=no, name=Correct Ans]{cor}

\declaretheoremstyle[
  headfont=\bfseries\sffamily\color{black}, bodyfont=\normalfont,
  mdframed={
    linewidth=2pt,
    linecolor=black,
    nobreak=false
  }
]{thmblackline}

\declaretheorem[style=thmblackline, numbered=no, name=Description]{problem}


\usepackage{titlesec}

\titleformat{\section}
  {\normalfont\Large\bfseries}  % 字型樣式
  {}                            % 不顯示編號
  {0pt}                         % 編號和標題距離
  {Problem\ }                  % 每個 section 都會以 "Problem:" 開頭

\newcommand{\red}[1]{\textcolor{red}{#1}}
\newcommand{\blue}[1]{\textcolor{blue!70!black}{#1}}
\newcommand{\yel}[1]{\textcolor{Goldenrod!70!black}{#1}}
\newcommand{\grn}[1]{\textcolor{ForestGreen!70!black}{#1}}

\newcommand{\yelbox}[1]{%
  {\setlength{\fboxrule}{1pt}%
   \setlength{\fboxsep}{2pt}%
   \color{Goldenrod!90!black}\fbox{\normalcolor #1}}}

\newcommand{\bluebox}[1]{%
  {\setlength{\fboxrule}{1pt}%
   \setlength{\fboxsep}{2pt}%
   \color{cyan!80!black}\fbox{\normalcolor #1}}}

\newcommand{\redbox}[1]{%
  {\setlength{\fboxrule}{1pt}%
   \setlength{\fboxsep}{2pt}%
   \color{red}\fbox{\normalcolor #1}}}

\usepackage{pgfplots}
\pgfplotsset{compat=1.18}


\usepackage{adjustbox}
\usepackage{centernot}

\title{MATH-4007 Calculus 2 class 14 , Homework 9 常見錯誤}
\author{B13902126 胡允升}
\date{}  
\setlength{\parindent}{0pt}


\begin{document}
\fontsize{12pt}{16pt}\selectfont

\maketitle  
\noindent\rule{\linewidth}{0.4pt}
\textbf{For the writing style}:

\noindent\rule{\linewidth}{0.4pt}
\textbf{再再再再再再次提醒拜託各位寫題目的時候}
\begin{itemize}
    \item 如果是用電子檔寫題目可以「新開一頁」,把答案寫在下一頁
    \item 如果是用紙本,可以拿一張新的紙把題目標清楚
    \item 標示清楚計算過程
\end{itemize}
以減少助教眼壓。

\noindent\rule{\linewidth}{0.4pt}

\vspace{2em}

\noindent\rule{\linewidth}{0.4pt}
\textbf{For studying(關於課程網上的影片)}:

\noindent\rule{\linewidth}{0.4pt}

教授上課不一定能夠 cover 到所有題目,助教們在 NTUCOOL 上都會放上「詳解」影片,\textbf{請務必要觀看},在這次題目中,有不少題都是教授上課沒有 cover 到的講義內容變化題,明顯的很多人並沒有觀看助教們錄好的影片,希望大家能夠確實觀看,才能在期考拿下高分。

\noindent\rule{\linewidth}{0.4pt}

\newpage

\section{2-(b)}

\begin{problem}
Find the following the indefinite integral
\[
	\int \left( \frac{1}{2x} + \frac{x-1}{\sqrt{x}} \right)
\]
\end{problem}
\vspace{-12pt}
\begin{cor}
	\[ 
		\frac{1}{2} \ln|x| + \frac{2}{3} x^{\frac{3}{2}} - 2 x^{\frac{1}{2}} + C
	\]
\end{cor}
\vspace{-12pt}
\begin{wrong}
	\[ 
		\frac{1}{2} \ln x + \frac{2}{3} x^{\frac{3}{2}} - 2 x^{\frac{1}{2}} + C
	\]
\end{wrong}

\begin{reason}
	See that the integrand can be rewritten as
	\begin{align*}
		\int \left( \frac{1}{2x} + \frac{x-1}{\sqrt{x}} \right) &= \int \left( \frac{1}{2x} + \sqrt{x} - \frac{1}{\sqrt{x}} \right) \\[8pt]
		&= \int \frac{1}{2x} \ \text{d} x + \int \sqrt{x} \ \text{d} x - \int \frac{1}{\sqrt{x}} \ \text{d} x \\[8pt]
		&= \frac{1}{2} \ln|x| + \frac{2}{3} x^{\frac{3}{2}} - 2 x^{\frac{1}{2}} + C	
	\end{align*}

	\begin{note}
		注意到 $\ln x$ 的定義域為 $(0, \infty)$,而題目中並沒有特別限制 $x$ 的範圍,因此應該使用 $$\boxed{\ln|x|}$$ 來涵蓋 $x < 0$ 的情況。
	\end{note}
\end{reason}


\newpage

\section{4-(a)-ii}

\begin{problem}
Find the answer by Fundamental Theorem of Calculus
\[
	f(x) = \int_{x^2}^{e^x} \sqrt{t} \sin(t) \ \text{d} t
\]
find $f'(x)$.
\end{problem}
\vspace{-12pt}
\begin{cor}
	\[ 
		f'(x) = \sqrt{e^x} \sin(e^x) \cdot e^x - |x| \sin(x^2) \cdot (2x)
	\]
\end{cor}
\vspace{-12pt}
\begin{wrong}
	\[ 
		f'(x) = \sqrt{e^x} \sin(e^x) \cdot e^x - \sin(x^2) \cdot (2x^2)
	\]
\end{wrong}

\begin{reason}
	Since,
	\[
		\int_{x^2}^{e^x} \sqrt{t} \sin(t) \ \text{d} t = \int_{0}^{e^x} \sqrt{t} \sin(t) \ \text{d} t - \int_{0}^{x^2} \sqrt{t} \sin(t) \ \text{d} t
	\]
	我們可以簡化式子
	\begin{align*}
		f'(x) &= \left( \int_{x^2}^{e^x} \sqrt{t} \sin(t) \ \text{d} t \right)' \\
			  &= \left( \int_{0}^{e^x} \sqrt{t} \sin(t) \ \text{d} t - \int_{0}^{x^2} \sqrt{t} \sin(t) \ \text{d} t \right)' \\
			  &= \left( \int_{0}^{e^x} \sqrt{t} \sin(t) \ \text{d} t \right)' - \left( \int_{0}^{x^2} \sqrt{t} \sin(t) \ \text{d} t \right)'
	\end{align*}
	By Fundamental Theorem of Calculus
	\[
		f'(x) = \underset{\int_{0}^{e^x} \sqrt{t} \sin(t) \ \text{d} t}{\underbrace{\sqrt{e^x} \sin(e^x) \cdot e^x}} - \underset{\int_{0}^{x^2} \sqrt{t} \sin(t) \ \text{d} t}{\underbrace{|x| \sin(x^2) \cdot (2x)}}
	\]

	\begin{note}
		注意到 
		\[
			\left( \int_{0}^{x^2} \sqrt{t} \sin(t) \ \text{d} t \right)' = \begin{cases}
				x \sin(x^2) \cdot (2x) & x \geq 0 \\
				-x \sin(x^2) \cdot (2x) & x < 0
			\end{cases}
		\]
		因此,若要寫成一個式子,需要加上絕對值符號:
		\[
			\boxed{|x| \sin(x^2) \cdot (2x)} 
		\]

	\end{note}
\end{reason}

\newpage


\end{document}